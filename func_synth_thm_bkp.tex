\section{Synthesizing Rectification Functions}\label{comp:synth}

The above techniques show how to construct a rectification function
by reasoning about the varieties of $rem_l$.  
% The techniques described above evaluate an unknown rectification
% function $u_i$ by reasoning about its variety and the variety of the remainders $rem_l$.
However, algebraically, we compute these functions using their corresponding ideals.
Specifically, we show how the remainders computed in Theorem~\ref{Thm:rect} can be utilized for rectification function computation.
For this purpose, we restate Theorem 8.5 and its proof from~\cite{UTK:thesis}, which we utilize. 

% Since it is not practical to compute varieties, we provide a procedure to
% obtain the rectification functions from the generators of these ideals themselves. 

\begin{Theorem}
\label{thm:coffF2}
Given a polynomial ring $\Fn[x_1,\dots,x_d]$ and a variety $V \subseteq \mathbb{F}_2^d$,
the ideal $\mathcal{I}_{\Fn}(V)$ that vanishes on $V$ and has coefficients in $\Fn$ 
can be expressed as the product $A \cdot\Fn[x_1,\ldots,x_d]$, where $A = \bigcap_{a\in V}m_a$
and $m_a = \langle x_1-a_1,\dots,x_d-a_d \rangle \subseteq \F_2[x_1, \ldots, x_d]$ for 
each $a=(a_1,\dots,a_d) \in V$.
\end{Theorem}

\begin{proof}
Consider a single point $a \in V$ and the ideal $\mathcal{I}_{\Fn}(\{a\})$ which has coefficients in 
$\Fn$ and vanishes on $a$. This ideal can be written as,
\begin{align*}
\mathcal{I}_{\Fn}(\{a\})=\mathcal{I}_{\overline{\Fn}}(\{a\})\cap\Fn[x_1,\dots,x_d]
\end{align*}
The above expression can further simplified as follows,
\begin{align*}
\mathcal{I}_{\overline{\Fn}}(\{a\})\cap \Fn[x_1,\ldots,x_d] 
&= (m_a\cdot \overline{\Fn}[x_1,\ldots,x_d])\cap\Fn[x_1,\ldots,x_d] \\
&= m_a\cdot \Fn[x_1,\ldots,x_d]
\end{align*}

Now consider the ideal $\mathcal{I}_{\Fn}(V)$ which can written as the intersection of 
$\mathcal{I}_{\Fn}(\{a\})$ for each $a \in V$,
\begin{align*}
\mathcal{I}_{\Fn}(V) &= \bigcap_{a\in V}\mathcal{I}_{\Fn}(\{a\}) \\
&= \bigcap_{a\in V}m_a\cdot \Fn[x_1,\ldots,x_d] \\
&= (\bigcap_{a\in V}m_a)\cdot \Fn[x_1,\ldots,x_d],
\end{align*}
where the last equality is due to $\F_2[x_1,\ldots,x_d] \subseteq \Fn[x_1,\ldots,x_d]$.
\end{proof}

Now consider the remainders $rem_l$ ($\langle rem_l \rangle \subseteq \Fn[X_{PI}] $) computed in Theorem~\ref{Thm:rect}. 
Even though the remainders $rem_l$ have coefficients in $\Fkn$ (higher field), their varieties 
are in $\F_{2}^{|X_{PI}|}$ ($V(\langle rem_l\rangle)\subseteq \F_2^{|X_{PI}|}$) as they correspond to 
bit-level assignments to $X_{PI}$.
We have that $\mathcal{I}_{\Fn}(V(\langle rem_l\rangle)) = \langle rem_l \rangle + J_0$ (Theorem \ref{thm:strong-ns}).
It is a property of such ideals ($\langle rem_l \rangle + J_0 $) that the generators of their reduced
\Grobner bases (Def.~\ref{def:rgb}) have coefficients only in $\Ftwo$. Let $G_l=redGB(\langle rem_l \rangle + J_0 )$
denote the generators of the reduced \Grobner basis computation.
Then using Theorem~\ref{thm:coffF2}, $G_l = A \cdot \Fn[\xpi]$ where $A = \bigcap_{a\in V(G_l)}m_a$
and $m_a$ is as defined in the theorem statement. 
% Since $G_l$ is computed using the Theorem~\ref{thm:elm},
% it is a Gr\"obner basis itself, and therefore, $E_L = A = \bigcap_{a\in \vpi(E_L)}m_a \subseteq \F_2[\xpi]$.
% In other words, the generators of the ideal $E_L$ computed using the Theorem~\ref{thm:chk} have coefficients in $\F_2$.
% The same result can also be shown for the generators of the ideals $E_H$ and $E'_H$. 
% Consequently, any ideal-interpolant $J_I$ computed
% for the pair $(E_L,E_H)$ will also have generators in $\F_2[\xpi]$. 
This, in turn implies that for 
any point in $\F_2^{|\xpi|}$, each generator of $G_l$ will either evaluate to 0 or 1.
Using these results, a rectification polynomial $\uc$ can be computed
from the generators of $G_l$ and subsequently, a sub-circuit can be synthesized from $\uc$. 


% {\color{red}
% The ideal-interpolant $J_I$ is computed through a \Grobner basis which
% is a set of polynomials $G_I = \{g_1,\dots, g_t\}$. It is non-trivial
% to synthesize a single-output Boolean function from this set of
% polynomials. This problem can be simplified if we can compute a
% rectification  $\uc$ from $G_I$ with $V(\uc) = V(G_{I})$,
% and then synthesize it into a subcircuit.  
% %More specifically, we use the generators of $E_L$ for the computation. 
% %As our goal is to synthesize the rectification function $\uc$, 
% This can be achieved if the coefficients of $\uc$ are in
% $\F_2=\{0,1\}$, as $X_{PI}$ variables already take values in
% $\{0,1\}$. {\bf For this we first (wish to) prove the following
%   conjecture on the nature of generators of the ideal $E_L$.}

% \begin{Conjecture}
% \label{thm:ELF2}
% Ideal $E_L$ as given in Thm \ref{thm:chk} is computed as follows:
% \begin{enumerate}
% \item Compute $G=GB(J_L+J_0)$ with the lex order ``non-$\xpi$ variables $> \xpi$ variables''
% \item From $G$, obtain $G_L = G \cap\Fkk[X_{PI}] = \{g_1,\dots,g_t\}$
%   as the \Grobner basis of the elimination ideal $E_L$, i.e. $G_L = GB(E_L)$.
% \end{enumerate}
% Then the polynomials $g_1,\dots,g_t \in \F_2[\xpi]$.
% % Then from this $GB$ set, we pick the polynomials that contain only $\xpi$ variables.
% % We know that $\vpi(E_L) \subseteq \F_2^{|\xpi|}$, can we show that the 
% % coefficients of the generators of the ideal $E_L$, computed as above,
% % are from the set $\{0,1\}$. 
% \end{Conjecture} 

% {\bf
% Florian:  If we can show that the above conjecture is true, $\uc$ computed
%   from Eqn. \ref{eqn:id2poly} will evaluate to $\{0,1\}$, i.e in $\F_2$.
% As a result, $\uc$ can be synthesized using AND and XOR gates. In all
% our experiments, we have observed that the computed \Grobner basis is
% in $\F_2[\xpi]$. In subsection \ref{sec:conj_proof}, Utkarsh gives a
% reasoning for why we think this always happens, and his reasoning relies on
% the subsequent concepts. That is why it is described in two sections.
% A similar proof can also be shown for the generators of $E_H$ and
% $E'_H$. 
% }}

\subsubsection{Obtaining $\uc$ from $E_L$}

%% In finite fields, given an ideal $J$, it always possible to find a 
%% polynomial $U$ such that $V(U) = V(J)$. The reason is that
%% every ideal in a finite field has a finite
%% variety and a polynomial with those points as its roots can always be
%% constructed using the Lagrangian interpolation formula. We construct
%% the rectification polynomial $U$ from the ideal-interpolant $E_L$ as
%% shown below, such that $V(E_L) = V(U)$. 

With $GB(E_L) = \{g_1,\dots,g_t\}$, $U$ is computed as:
\begin{equation}
\label{eqn:id2poly}
U = (1+g_1)(1+g_2)\cdots(1+g_t)+1
\end{equation}

It is easy to assert that $V(U) = V(E_L)$. 
Consider a point $\bm{a} \in V(E_L)$. As all of 
$g_1,\dots,g_t$ vanish (= 0) at $\bm{a}$,
\begin{align*}
U(\bm{a}) &= (1+g_1(\bm{a}))(1+g_2(\bm{a}))\cdots(1+g_t(\bm{a}))+1 \\
&= (1+0)(1+0)\cdots(1+0) + 1 = 0
\end{align*}
Conversely, for a point $\bm{a'} \not \in V(E_L)$, at least one  
of $g_1,\dots,g_t$ will evaluate to 1. 
Without loss of generality, if
$g_1$ evaluates to 1 at $\bm{a'}$, then $U=(1+1)(1+0)\cdots(1+0)+1 = 1 \neq
0$. In addition to $\vpi(\uc) = \vpi(E_L)$, $\uc$ either evaluates to 0 or 1. 
%%%%%%%%% The algorithm is commented out
%% \par Using Eqn. (\ref{eqn:id2poly}), a recursive procedure is derived to
%% compute $U$, and it is depicted in Algorithm \ref{algo:id2poly}. At
%% every recursive step, we also reduce the intermediate  
%% results by $\pmod{ J_0}$ (line 7) so as to avoid terms of high degree.
In this fashion, from the ideal-interpolant $E_L$, we compute the
single-fix rectification polynomial function $\uc$, and synthesize a
sub-circuit at net $x_i$ such that $x_i = \uc$ rectifies the circuit. 


% However, in~\cite{Utkarsh:VLSI18}, it was 
% shown that it is a property of such ideals ($\langle rem_l, J_0 \rangle$) that their reduced
% \Grobner bases (Def.~\ref{def:rgb}) have coefficients only in $\Ftwo$.
Further, it was shown that, given an ideal $I$ with coefficients in $\Ftwo$ with generators  $\{g_1,\dots,g_t\}$, 
a polynomial $p$ can always be constructed as $p = (1+g_1)(1+g_2)\dots(1+g_t)+1$, such that $V(p) = V(I)$. 
% This helps in computing a singleton polynomial for each 
%% {\it Ideal-Poly Conversion:} Over finite fields, given an ideal $J$, 
%% a polynomial $p$ can always be computed 
%% such that $V(p) = V(J)$. Let $\{g_1,\dots,g_t\}$ denote the generators
%% of $J$. Then, the polynomial $p$ can be constructed as, 
%% $p = (1+g_1)(1+g_2)\dots(1+g_t)+1$~\cite{Utkarsh:VLSI18}. 
% i.e. $V(rem_l)\subseteq \F_{2}^{|X_{PI}|}, 1 \leq l \leq 2^m$.
% It is a property of such ideals ($\langle rem_l,\jzxpi \rangle$), that their Reduced
% \Grobner basis (Def.~\ref{def:rgb}) will have coefficients only in $\Ftwo$.
% Once we compute a GB with coefficients in $\Ftwo$, we can translate it to a polynomial 
% which has the same variety as the ideal (Sec.~\ref{sec:prelim}).
Consequently, the rectification function operations are restricted to
algebraic computations in $\F_2[X_{PI}]\equiv \B$. 

To compute the patch $u_i$, we perform the following steps:
\bi
\item Compute reduced \Grobner bases of $\langle rem_l, J_0 \rangle$.
\item Construct a singleton polynomial $p$ such that \\ $V(p) = V(\langle rem_l, J_0 \rangle)$.
\item Impose an order on the sets for $DC_{pair}$ and composite set computations.
\item Compute $DC_{pair}$ using Eqn.~(\ref{eqn:dc_pair}), and then obtain the composite sets
      in Eqn.~(\ref{eqn:composite_dc}) which are then assigned to DC-, on- and off-sets of the 
      rectification functions (Sec.~\ref{comp:DFC1}).
\item In order to perform the variety union, intersection, and difference operations, we use ideal 
        product, sum, and colon operations, respectively, on the
        singleton polynomial $p$ representation of the ideal.
% Use ideal operations sum, product, and colon ideal to perform intersection, 
% sum, and difference of varieties, respectively.
\item The above procedure delivers $u_{i_{DC}}$ and $u_{i_{on}}$ as singleton polynomials in $\Ftwo[X_{PI}]$.
Translate $u_{i_{DC}}$ and $u_{i_{on}}$ into Boolean functions by interpreting the product, sum,
and '+1' as Boolean AND, XOR, and INV gates, respectively. Optimize the on-set $u_{i_{on}}$ w.r.t. to the DC-set $u_{i_{DC}}$.
\ei

%  A polynomial in $\F_2[X_{PI}]$ can be converted to a Boolean AND-XOR operation. This expression can be
% synthesized into a sub-circuit whose output can be connected to the respective targets in $W$ to rectify the circuit.
% As the product and sum operations are performed modulo 2; they can be implemented
% in a circuit using AND and XOR gates, respectively.

% The expression can be synthesized into a sub-circuit by replacing the modulo 2 product and sum 
% in the polynomial expression with the Boolean AND and XOR operators, respectively.
% We synthesize a subcircuit by interpreting sum as XOR $\pmod{2}$ and product 
% as AND gates.
% Subsequently,  we convert the polynomial to corresponding Boolean functions by treating $+,\cdot,+1$ 
% as XOR, AND, and INV operations, respectively.

% In order to compute a desired $u_i \in \F_2[X_{PI}]$ with a 
% $\F_2^{|X_{PI}|} \mapsto \F_2$ mapping,
% we perform the following steps.


% \bi
% \item Compute the generators $Gr_l$ for the ideal $Jr_l=\langle rem_l \rangle$ as 
% $redGB(Jr_l+\jzxpi)$, where $rem_l \in \Fkn[X_{PI}]$.
% Then the polynomials in $Gr_l$ have coefficients in $\F_2$,
% and variables in $\xpi$[Corollary~8.1]~\cite{UTK:thesis}.
% \item Use the Ideal-Poly conversion procedure from (Section \ref{sec:prelim})
% to compute a polynomial $rem_l'$. The computed $rem_l'$ is such that 
% $rem_l'\in \F_2[X_{PI}]$ and $V(rem_l') = V(rem_l)$.
% \item To compute $u_i$ algebraically, we utilize the relevant ideal operations described in
% the ideal-variety correspondences (Section \ref{sec:prelim}) to perform the union, intersection,
% and set difference of varieties over $V(rem_l')$.
% \ei


% $U_{dc} = \bigcap\limits_{l=1}^{2^m}V(pr_l)$

% \begin{Example}
% Continuing with our example~\ref{ex:4}, for the circuit shown in Fig.~\ref{fig:mas_bug_W},
% the intersection of Variety of remainders yields us the DC for all the targets.

% $U_{dc} = \bigcap\limits_{l=1}^{4}V(pr_l) = a_2b_1b_2+a_2b_1+a_2b_2+1$
% \end{Example}

% However, our experiments show that for a large class of benchmarks depending on the
% target selection, the intersection of all sets is most likely to be empty. Thus,
% we need a stronger DC set computation for individual targets.

% Recall that the variety of each $pr_l$ corresponds to the set of correct points
% for a given tuple assignment to the targets from $W_c[l]$. Thus, all 
% the points which lie in the intersection of all the $pr_l$'s comprises the 
% DC for all the targets.
% We can compute a polynomial $pr_l \in \F_{2}^{|X_{PI}|}$
%  such that $V(pr_l) = V(rem_l)$~\cite{Utkarsh:VLSI18}. 
% \subsubsection{DC for individual targets}

\begin{Example}\label{ex:4}
  Continuing with Ex.~\ref{ex:3}:
  % , we illustrate the above procedure:

  \bi
  \item $rem_3 = (\ga+1)a_2b_1b_2+a_2b_1 + (\ga^2+\ga)a_2b_2$ %\\ coefficients in $\Fkn$
  \item $redGB(\langle rem_3,J_0\rangle) = \{a_2b_1,a_2b_2\}$ %\\ coefficients in $\Ftwo$
  \item $p_{rem_3} = (1+a_2b_1)*(1+a_2b_2)+1 = a_2b_1b_2+a_2b_1+a_2b_2$
  \bi
    \item Here, $V(p_{rem_3}) = V(\langle rem_3, J_0 \rangle)$
  \ei
  \item Similarly:
  \bi
    \item $p_{rem_1} = a_2b_2$
    \item $p_{rem_2} = a_2b_1b_2$
    \item $p_{rem_4} = a_2b_1b_2 + a_2b_1$
  \ei
  \ei
  The rectification polynomials for the targets $(r_3,rr_3)$ computed using the greedy approach~(Sec.~\ref{comp:GFC})
  \begin{align*}
    % \begin{split}
      &u_{1_{on}}  = a_2b_1b_2; &u_{1_{off}} &= a_2b_1b_2 +1; &r_3  &= u_{1_{patch}} =(a_2\wedge b_1 \wedge b_2);\\
      &u_{2_{on}}  = a_2b_2;    &u_{2_{off}} &= a_2b_2+1;     &rr_3 &= u_{2_{patch}}= (a_2\wedge b_2).
    % \end{split}
  \end{align*}
  The rectification polynomials for the targets $(r_3,rr_3)$ computed using the on-set and don't care simplification~(Sec.~\ref{comp:DFC2})
  \begin{flalign*}
    % \begin{split}
     &u_{1_{dc}} = a_2b_1b_2+a_2b_2;&u_{1_{on}} &= a_2b_1b_2; &r_3 &= u_{1_{patch}} = a_2\wedge b_2; \\
     &u_{2_{dc}} = a_2b_2+1;        &u_{2_{on}} &= a_2b_2;    &rr_3 &= u_{2_{patch}} = 1. 
    % \end{split}
  \end{flalign*}
  % As seen above, the synthesis tool ({\it sis}) was able to simplify 
  % the on-set for $u_1$ patch function.
\end{Example}

\section{Improving efficiency using ZDDs}\label{comp:synth}

The techniques described in rectification check and function computation 
predominantly involve \Grobner basis computations and polynomial reductions.
By virtue of RTTO $>$, the complexity of remainder ($rem_l$) generation 
in the rectification check was moved from one of computing GB to that of 
GB-reduction by way of multivariate polynomial division. 
However, the function computation operation also involves computing
a reduced GB, which can be especially prohibitive when applied to circuits with
large size operands. Hence, there is a need for efficient representation 
to overcome the infeasibility of these operations using conventional 
computer algebra tools. 
 
 to overcome
this complexity. As the GB algorithms are employed to compute rectification function
patches via CI on the circuit, we analyze the circuit and its topology to derive specific
information, which we use to guide the computation of patches efficiently. Moreover, we
further show how our ZDD based algorithms can effectively utilize this information
to compute rectification patches. By combining our theories and algorithms, we can explore
the space of all possible rectification functions through their ON-set, OFF-set, and DC-set
of corresponding Boolean functions.
