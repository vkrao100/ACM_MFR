\section{Efficiency improvement using ZDDs}\label{impl}

The techniques described in rectification check and function computation 
predominantly involve \Grobner basis computations and polynomial reductions.
By virtue of RTTO $>$, the complexity of remainder ($rem_l$) generation 
in the rectification check was moved from one of computing GB to that of 
GB-reduction by way of multivariate polynomial division. 
However, the function computation operation also involves computing
a reduced GB, which can be especially prohibitive when applied to circuits with
large size operands. Hence, there is a need for efficient representation 
to overcome the infeasibility of these operations using conventional 
computer algebra tools. 
 
 As the GB algorithms are employed to compute rectification function
patches via remainders generated from the circuit, we analyze the circuit and its 
topology to derive specific
information, which we use to guide the computation of patches efficiently. Moreover, we
further show how our ZDD based algorithms can effectively utilize this information
to compute rectification patches. By combining our theories and algorithms, we can explore
the space of all possible rectification functions through their ON-set, OFF-set, and DC-set
of corresponding Boolean functions.

In this regards, we take inspiration from~\cite{Utkarsh:TCAD19} and utilize 
PolyBori’s~\cite{pbori:JSC09} reduction procedure with 
ZDDs~\cite{Minato:DAC93,Minato:DAC94} as the underlying data structure 
to improve our proposed approach.
PolyBori proposed the use of ZDDs to compute
\Grobner bases for Boolean polynomials. PolyBori is a generic Boolean GB computational
engine that caters to many permissible term orders. Its division algorithm is also generally
based on the conventional concept of canceling one monomial in every step of reduction.
In contrast, our algorithms are tailored for GB-reduction under the RTTO > . The efficiency
of our approach stems from the observation that the RTTO > imposes a special structure
on the ZDDs, which allows for multiple monomials to be canceled in one division-step,
along with simplifying the search for divisors.
The efficiency is derived by treating the polynomials 
as unate cube sets and checking isomorphism between the implementation and specification 
graphs. 

% We reason about the presence or absence of solutions and other properties of a 
% system of polynomials without explicitly solving them. 
% The Boolean values of the nets of a circuit for all possible input assignments is a set 
% of points, which can be construed as solutions to a set of polynomials.
% We further improve upon the rectification approach by exploiting the circuit 
% topology and ZDD based GB reductions.