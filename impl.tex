\section{Efficient Implementation of MFR setup using ZDDs}\label{impl}

Taking inspiration from~\cite{Utkarsh:TCAD19}, we utilize 
PolyBori’s~\cite{pbori:JSC09} reduction procedure with 
ZDDs~\cite{Minato:DAC93,Minato:DAC94} as the underlying data structure 
to improve our proposed approach.
PolyBori proposed the use of ZDDs to compute
\Grobner bases for Boolean polynomials. PolyBori is a generic Boolean GB computational
engine that caters to many permissible term orders. Its division algorithm is also generally
based on the conventional concept of canceling one monomial in every step of reduction.
In contrast, our algorithms are tailored for GB-reduction under the RTTO > . The efficiency
of our approach stems from the observation that the RTTO > imposes a special structure
on the ZDDs, which allows for multiple monomials to be canceled in one division-step,
along with simplifying the search for divisors.
The efficiency is derived by treating the polynomials 
as unate cube sets and checking isomorphism between the implementation and specification 
graphs. We reason about the presence or absence of solutions and other properties of a 
system of polynomials without explicitly solving them. 
The Boolean values of the nets of a circuit for all possible input assignments is a set 
of points, which can be construed as solutions to a set of polynomials.
We further improve upon the rectification approach by exploiting the circuit 
topology and ZDD based GB reductions.