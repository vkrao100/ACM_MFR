\section{Preliminaries}
\label{sec:prelim}
This section reviews some basic concepts from symbolic computer
algebra and associated algorithms utilized in this article. \\

%\subsection{Notation and Background}
\bif{Finite fields:} Let $\B$ denote the Boolean domain, and $\neg,
\wedge,\vee, \oplus$ the NOT, AND, OR and XOR operators,
respectively. Let $\F_2=\{0,1\}$ be the field of 2 elements, and let
$\F_q = \Fkn$ denote the {\it finite field} of $q=2^n$ elements, for a
given $n\in \Z_{> 0}$. We denote $\Fqbar$ as the {\it algebraic closure} of $\Fq$. 
The algebraic closure of the finite field $\Fkn$ is the union of all fields $\Fkk$
such that $n | k$ ($n$ divides $k$).
$\Fn$ is the $n$-dimensional extension of
$\F_2$, and it is constructed as $\Fn = \F_2[x]\pmod{ P_n(x)}$. Here
$P_n(x) \in \F_2[x]$ is a given degree-$n$ polynomial, {\it
  irreducible} in $\F_2$, with a root $\gamma$, i.e.  $P_n(\gamma)=0$ 
 , which is a generator of
$\Fn$. Then $\gamma$ is called a primitive element (PE) of $\Fn$ and
it generates the entire field: $\Fn=\{0,1=\gamma^{2^n - 1}, \gamma,
\gamma^2,\dots,\gamma^{2^n - 2}\}$. Note: $\forall \gamma\in \Fkn, 
\gamma^{2^n-1}=1$.
An element $A \in \Fn$ can be written as $A = a_0 + a_1\cdot \gamma +
\dots + a_{n-1}\cdot\gamma^{n-1}$, where $a_0,\dots,a_{n-1} \in
\F_2$. 
In $\Fn$, the addition ($''+''$) and multiplication
($''\cdot''$) operations are performed in the base field $\F_2$ and
reduced modulo the corresponding primitive polynomial $P_n(x)$. For
$n, k \in \Z_{> 0}$, if $n | k$, then $\Fn \subset
\Fkk$.
Thus, $\F_2 \subset \F_{2^n}, \forall n>1$. 
All fields of the type
$\Fkn$ have characteristic 2, and therefore $-1 = +1$ in $\Fkn$. \\


\bif {Polynomial representation:} Let $R=\F_q[x_1,\dots,x_d]$ be the
polynomial ring in variables $x_1,\dots,x_d$ with coefficients in
$\F_q$. A polynomial $f \in R$ is 
written as a finite sum of terms  $f = c_1 M_1 +  c_2 M_2 + \dots +
c_r M_r$, where $c_1, \dots, c_r$ are coefficients from $\F_q$ and
$M_1, \dots, M_r$ are monomials, i.e. power products of the type
$x_1^{e_{1}}\cdot x_2^{e_{2}}\cdots x_d^{e_{d}}$,  $e_j \in \Z_{\geq  0}$. 
 To systematically manipulate the polynomials, a monomial order $>$ (or
a term order) is imposed on $R$ such that the monomials of all
polynomials are ordered according to $>$. 
% A monomial order $>$ (term order) is always imposed on $R$, and
Subject to $>$, we have that $M_1 >M_2 > \dots > M_r$. Then, $lc(f) = c_1$, $lm(f) =
M_1$, and $lt(f) = c_1 M_1$ denote the {\it leading coefficient, leading monomial}, 
and {\it leading   term} of $f$, respectively. 
 Also, for $f\in R$, $tail(f) = f - lt(f)$.
This work employs {\it lexicographic} (lex) term orders.  \\

%%%%%%%%%%%%%%%%%%%% PK
%% A circuit with $n$-bit operands is modeled as a polynomial function in
%% $\Fn$, where variables $x_1,\dots,x_d$ denote the nets of the circuit.
%% %%%% Move to verification
%% Logic gates of the circuit can be modeled with polynomials in
%% $\F_2 \subset \Fn$, with the mapping $\B \mapsto \F_2$:
%% %% As $\Fkk \supset \F_2$, these polynomials can also be
%% %% construed as polynomials in $\Fkk$.
%% %% given as: 
%% \begin{equation}
%% \label{bool2poly}
%% \begin{split}
%% z ~ =  ~ \neg a ~ \rightarrow ~ z+a+1 & \pmod 2  \\
%% z ~ =  ~ a \wedge b ~ \rightarrow ~ z+a \cdot b & \pmod 2\\
%% z ~ =  ~ a \vee b ~ \rightarrow ~ z+a+b+a \cdot b & \pmod 2 \\
%% z ~ =  ~ a \oplus b ~ \rightarrow ~ z+a+b & \pmod 2 
%% \end{split}
%% \end{equation}

\bif{Polynomial Reduction via division:} 
Let $f,~g$ be polynomials. If
$lm(f)$ is divisible by $lm(g)$, then we say that $f$ {\it is
reducible to} $r$ modulo $g$, denoted $f
\stackrel{g}{\textstyle\longrightarrow} r$, where $r = f - \frac{lt(f)}{lt(g)} 
\cdot g$. This operation forms the core  of polynomial
division algorithms and it has the effect of canceling the 
leading term of $f$ to obtain $r$. Similarly, 
$f$ can be {\it reduced} w.r.t a set of polynomials $F=\{f_1,\dots,f_s\}$. 
Then $f\xrightarrow{F}_+r$ denotes the {\it
  reduction} of $f$ modulo the set of polynomials $F$ resulting in a
remainder $r$, obtained by iteratively canceling terms in $f$ by
$lt(f_j), f_j\in F$, via polynomial division. 
 The remainder $r$ is said to be {\it reduced} such that no term in $r$ is
divisible by the leading term of any $f_j \in F$.

Algorithm~\ref{algo:mv_reduce}~\cite{gb_book} presents the procedure which
performs polynomial {\it reduction} by iteratively canceling
one monomial at a time.
The algorithm initializes $h$ with the polynomial $f$ and cancels its
leading term by some polynomial $f_j$. If the leading term $lt(h)$
cannot be canceled by any $lt(f_j)$, then it is added to the final
remainder $r$. We repeat this process until all the terms in $h$
are analyzed.  
%% We represent the given circuit by way of a set of polynomials
%% $F=\{f_1,\dots,f_s\}$, and for verification and rectification, we
%% analyze the solutions to the polynomial equations $f_1=\dots=f_s=0$. 
%% For this purpose, we
%% consider the {\it ideal} generated by the polynomials, and their {\it
%%   variety.} 
Along with the remainder $r$, the algorithm also returns
the set of quotients $\{u_1,\dots,u_s\}$ of division of $f$ by
$\{f_1,\dots,f_s\}$, respectively, such that $f = u_1\cdot
f_1+\dots+u_s\cdot f_s + r$.

Section \ref{sec:pmodel} in the sequel describes how to represent a circuit $C$ as a set of 
polynomials $F=\{f_1,\dots,f_s\}$ contained in $R$. We analyze the solutions 
to the polynomial equations $f_1=\dots=f_s=0$ and the multivariate polynomial specification $f=0$ 
to formulate verification and rectification.
For this purpose, we consider the {\it ideal} generated by the polynomials and their {\it
  variety.} 
% The specification for $C$, denoted by $f$ is also modeled as a multivariate
% polynomial over $R$.

\begin{algorithm}[hbt]
 \caption{Multivariate Reduction of $f$ by $F=\{f_1,\dots,f_s\}$}
 \label{algo:mv_reduce}
 \begin{algorithmic}[1]
 % \Procedure{$multi\_variate\_division$}{$f, f_1, \dots, f_s \in \F[x_1, \dots, x_d], f_i\neq 0$}
 \Procedure{$multi\_variate\_division$}{$f,\{f_1,\dots,f_s\},f_j\neq0$}
 % \ENSURE $u_1,\dots, u_s, r$ s.t. $f = \sum f_i u_i+r$ where $r$ is
 % reduced w.r.t. $F = \{f_1,\dots, f_s\}$ and max($lp(u_1)lp(f_1), \dots, lp(u_s)lp(f_s), lp(r)$) = $lp(f)$
 \State $u_j \gets 0; ~r \gets 0, ~h \gets f $ 
 \While {  $h \neq 0$ }
 \If{ $\exists j$ s.t. $lm(f_j) ~|~ lm(h)$}
 \State choose $j$ least s.t. $lm(f_j) ~|~ lm(h)$
 \State $u_j = u_j + \frac{lt(h)}{lt(f_j)}$
 \State $h = h - \frac{lt(h)}{lt(f_j)} f_j$
 \Else
 \State $r = r+ lt(h)$
 \State $h = h - lt(h)$
 \EndIf
 \EndWhile
 \State \Return $(\{u_1,\dots,u_s\} , r)$
 \EndProcedure
 \end{algorithmic}
 \end{algorithm}


%% 

%\begin{Definition}$\bf{\left[Ideal\right]}$
\bif{ Ideals and varieties:} A given set of polynomials 
$F=\{f_1,\dots,f_s\}$ from $R$, generates the {\it ideal} $J = \langle
F \rangle \subseteq R$, defined as $J = \langle f_1,\dots,$ $ f_s \rangle = \{
h_1\cdot f_1 + \dots+h_s\cdot f_s~:~h_1,\dots,h_s\in R\}$. 
Polynomials $f_1,\dots,f_s$ form the {\it generators} or the {\it basis} of ideal $J$.


% {\it Ideals and Varieties:} A set of polynomials 
% $F=\{f_1,\dots,f_s\}$ from $R$, generates the {\bf ideal} $J = \langle
% F \rangle \subseteq R$, defined as $J = \langle f_1, \dots, f_s \rangle = \{
% h_1\cdot f_1 + \dots+h_s\cdot f_s:  h_1,\dots,h_s\in R\}$. 
% %% \vspace{-0.1in}
% %% \begin{small}
% %% \begin{equation}
% %% J = \langle f_1, \dots, f_s \rangle = \{ h_1\cdot f_1 + \dots+h_s\cdot
% %% f_s:  h_1,\dots,h_s\in R\}
% %% \end{equation}
% %% \end{small}
% %% \vspace{-0.1in}
% Polynomials $f_1,\dots,f_s$ form the basis or generators of ideal $J$.
%\end{Definition}

Let $\bm{a} = (a_1,\dots,a_d) \in \Fq^d$ be a point in the affine
space, and $f$ a polynomial in $R$. If $f(\bm{a}) = 0$, we say
that $f$ {\it vanishes} on $\bm{a}$, or $\bm{a}$ is a {\it zero} of the polynomial $f$.
 In this work, we have to analyze
the {\it set of all common zeros} of the polynomials of $F = 
\{f_1,\dots,f_s\}$  that lie within the field $\Fq$ -- i.e. the set of
all point $\bm{a} \in \Fq^d$ such that
$f_1(\bm{a})=\dots=f_s(\bm{a})=0$. This zero set is called the {\it 
  variety}, which depends not just on the given set of polynomials in
$F$, but rather on the ideal generated by polynomials. We denote it by
$V(J)$, where: $ V(J)= V_{\Fq}(J) = V_{\Fq}(\langle f_1,\dots,f_s \rangle) = \{\bm{a} 
\in \Fq^d: \forall f \in J, f(\bm{a}) = 0\}.$

Boolean functions comprise a finite set of points. Hence, we can reason about them as varieties. If a point is an element of a variety, then that point can be considered either an on-, off-, or DC-set minterm of a corresponding Boolean function. 
% We model rectification check and subsequent function computation by reasoning about its ON-set, OFF-set, and DC-set by means of varieties. 
% Explicitly computing the set of common zeros is practically infeasible. 
Algebraic geometry analyzes the variety of ideals without explicitly computing them, which is infeasible in practice.
Thus, to compute the union, intersection, and set difference of functions, operations are performed on the represented ideals by analyzing their corresponding varieties.
~\autoref{tab:var_ideal_corr} describes the correspondence between operations on Boolean functions, its variety interpretation, and the computation on ideals. We interpret the points within on-, off-, and DC-sets as varieties and utilize the correspondences from~\autoref{tab:var_ideal_corr} for rectification formulations. 

\def\arraystretch{2}
\begin{table}[hbt!]
  \centering
  {\scriptsize
  % \captionsetup{font=scriptsize}
  \caption{Correspondences between algebraic operations and Boolean operations. 
  Here, $J_1$ and $J_2$ correspond to the ideal generated by $\langle f \rangle$ and $\langle g \rangle$, respectively.
  }\label{tab:var_ideal_corr}
  \begin{tabular} 
    {!{\vrule width 1pt} c !{\vrule width 1pt} c !{\vrule width 1pt} c !{\vrule width 1pt}}  \noalign{\hrule height 1pt}
    {\it {Boolean Function}} & {\it{Variety interpretation}} & {\it{Ideal operation}} \\                                   \noalign{\hrule height 1pt}
    $f \lor g$              & $V(J_1) \cup V(J_2)$            & $J_1 * J_2$           \\ \hline
    $f \land g$              & $V(J_1) \cap V(J_2)$            & $J_1 + J_2$               \\ \hline
    $f - g$                 & $V(J_1) \setminus V(J_2)$       & $J_1 : J_2$               \\ \hline
    % $\neg f$                   & $\overline{V}(J_1)$            & $J_0 : J_1$               \\ \noalign{\hrule height 1pt}
  \end{tabular}}
\end{table}

Given two ideals $J_1 = \langle
f_1,\dots,f_s\rangle,$ $J_2=\langle h_1,\dots,h_r\rangle$, the {\it 
  sum of ideals} is denoted as $J_1 + J_2 = \langle
f_1,\dots,$ $f_s,h_1,\dots,h_r\rangle$, their {\it  product} is
given as $J_1 * J_2 = \langle f_i\cdot h_j: 1\leq i\leq s, 1\leq
j\leq r\rangle$, and a {\it  colon ideal} operation is defined as
$J_1:J_2 = \{f \in R\ :\ f\cdot g \in J_1, \forall g \in J_2\}$. 
Ideals and varieties are dual concepts:
$V(J_1 + J_2) = V(J_1) \cap V(J_2)$, whereas $V(J_1 * J_2) = V(J_1)
\cup V(J_2)$. A colon ideal operation corresponds the set difference of 
two varieties, $V(J_1:J_2)=V(J_1)
\setminus V(J_2)$. 
Modern computer algebra tools~\cite{DGPS_410,pbori:JSC09} implement these ideal operations, which we utilize.\\
% These ideal operations are implemented in computer algebra tools, which we utilize.\\
% Algebraic geometry analyzes ideals to reason about varieties without explicitly computing the varieties, which is infeasible in practice. 
% In the third column, $" \cdot "~,~ " +" ~,~" : " $ represent product, sum, and colon operations on ideals, respectively; these ideal operations are implemented in computer algebra tools, which we utilize.
% understood and correspond to the union, intersection, and difference of varieties, respectively.
% In this paper, we compute the union, intersection, and set difference of functions, represented by varieties, by performing corresponding operations on ideals. 


% {\it \Grobner Bases:} An ideal $J$ may have many different sets of 
% generators. %such that their varieties are the same.
% A \Grobner basis (GB) is one such generating set $G=\{g_1,\dots,g_t\}$
% with special properties that helps to solve many polynomial decision
% and quantification problems.  
 % that is a canonical representation of the ideal. 
\bif{ \Grobner Bases:}
An ideal may have many different sets of generators such that
their varieties are the same. A {\it \Grobner basis} (GB) is
one such generating set with special properties that allows 
to solve many polynomial decision and quantification problems. 
\begin{Definition}
\label{def:gb}
{\it  [\Grobner Basis]}~\cite{gb_book}: 
For a monomial ordering $>$, a set of non-zero polynomials $G =
\{g_1,g_2,\dots,g_t\}$ contained in an ideal $J$, is called a
\Grobner Basis (GB) of $J$ $\iff$
$\forall f \in J$, $f\xrightarrow{g_1,g_2,\dots,g_t}_+0$. 
%$i \in \{1,\cdots, t\}$ 
\end{Definition}
% The GB $G$ for an ideal $J$ can be computed using the  Buchberger's
% algorithm  %\cite{buchberger_thesis}. 
% (cf. Alg. 1.7.1 in~\cite{gb_book}), which
% takes as input a set of polynomials $F = \{f_1,\dots, f_s\}$ and
% computes its GB $G = \{g_1,\dots,g_t\}$, such that $J = \langle
% F\rangle = \langle G\rangle$. Moreover, $V(J) = V(F) = V(G)$. 
% The GB $G$ for an ideal $J$ can be computed using the  Buchberger's
% algorithm  %\cite{buchberger_thesis}. 
% (cf. Alg. 1.7.1 in~\cite{gb_book})
The GB $G$ for an ideal $J$ can be computed using Buchberger's
algorithm~\cite{buchberger_thesis}. 
The algorithm, shown in Alg. \ref{alg:gb}, takes the set of
polynomials $F = \{f_1,\dots, f_s\}$ as input and computes their
GB $G = \{g_1,g_2,\dots, g_t\}$, such that $J = \langle
F\rangle = \langle G\rangle$. Moreover, the polynomials of $F$ and $G$
have the same solution-set, i.e. $V(J) = V(\langle F \rangle) = V(\langle G \rangle)$. 

In the algorithm,

\begin{equation}
\label{spoly}
\begin{split}
Spoly(f_i,f_j) = \frac{L}{lt(f_i)}\cdot f_i - \frac{L}{lt(f_j)}\cdot f_j
\end{split}
\end{equation}
where $L = LCM(lm(f_i),lm(f_j))$, and $LCM$ denotes the least common
multiple. 

% A GB can be {\it reduced} to eliminate redundant polynomials 
% from the basis. A reduced GB is a canonical representation of the
% ideal.
\begin{algorithm}
\caption {Buchberger's Algorithm}
\label{alg:gb}
\begin{algorithmic}[1]
 \Require {$F = \{f_1, \dots, f_s\}$}
 \Ensure  {$G = \{g_1,\dots ,g_t\}$} 
  \State $G:= F$;
  \While{$G' \neq G$}
    \State $G' := G$
    \For {each pair $\{f_i, f_j\}, i \neq j$ in $G'$}
      \State $Spoly(f_i, f_j) \stackrel{G'}{\textstyle\longrightarrow}_+h$ 
      \If {$h \neq 0$} \State $G:= G \cup \{h\}$ \EndIf
    \EndFor
  \EndWhile
\end{algorithmic}
\end{algorithm}

\bif{ Ideal Membership Test:} The above definition inherently provides
a decision procedure to test for membership of a polynomial in an ideal:
$f$ is a member of an ideal $J$ {\it if and only if} 
% dividing $f$ by the \Grobner basis $G = GB(J)$ gives the remainder 0. 
$f\xrightarrow{GB(J)}_+0$.
When $f\notin J$,
then division by $GB(J)$ results in a non-zero remainder $r$ that is
unique/canonical. 

% A polynomial $f$ is a member of ideal $J$ iff division of $f$ by
% $GB(J)$ gives the remainder 0. If $f \notin J$,
% then division by $GB(J)$, $f\xrightarrow{GB(J)}_+r$, results in a
% non-zero remainder $r$ that is unique.
% A \Grobner basis can be computed using Buchberger's 
% algorithm (cf. Alg. 1.7.1 in~\cite{gb_book}). 
A \Grobner basis can be further reduced. A reduced \Grobner basis
of $J$ ($redGB(J)$) is a canonical representation of the ideal $J$.

\begin{Definition} \label{def:rgb}
{\it  [Reduced \Grobner Basis]}:
    A redGB for a polynomial ideal $J$ is 
    a GB $G=\{g_{1},\dots,g_{t}\}$ such that:
    \begin{itemize}
        \item $lc(g_{i})=1,\forall g_{i}\in G$
        \item $\forall g_{i} \in G$, no monomial of $g_{i}$ 
        lies in $\langle lt(G-\{g_{i}\})\rangle$
    \end{itemize}
\end{Definition}

%% {\it Ideal-Poly Conversion:} Over finite fields, given an ideal $J$, 
%% a polynomial $p$ can always be computed 
%% such that $V(p) = V(J)$. Let $\{g_1,\dots,g_t\}$ denote the generators
%% of $J$. Then, the polynomial $p$ can be constructed as, 
%% $p = (1+g_1)(1+g_2)\dots(1+g_t)+1$~\cite{Utkarsh:VLSI18}. 

% A reduced \Grobner basis
% is computed by first computing a minimal \Grobner basis, and then reducing it.
\bif{ Strong Nullstellensatz, Radical ideals, and Varieties:}
For any element $\varphi\in\Fq$, $\varphi^q=\varphi$ holds. Therefore,
the polynomial $x^q-x$ vanishes everywhere in $\Fq$, and we call it a
{\it vanishing polynomial}. We denote by $F_0 =
\{x_1^q-x_1,\dots,x_d^q-x_d\}$ the set of all vanishing polynomials in
$R$, and  $J_0 = \langle F_0 \rangle$ denotes the ideal of all
vanishing polynomials in $R$. 
% Then $V_{\Fq}(J_0) = \Fq^d$, and for any
% ideal $J$, we have that $V_{\Fq}(J)=V_{\Fq}(J+J_0)$.
Then, $V_{\Fq}(J_0) =
V_{\Fqbar}(J_0) = \Fq^d$. Moreover, given any ideal $J$, $V_{\Fq}(J) =
V_{\Fqbar}(J) \cap\Fq^d = V_{\Fqbar}(J) \cap V_{\Fqbar}(J_0) =
V_{\Fqbar}(J+J_0) = V_{\Fq}(J+J_0)$. 


\begin{Definition}{\it [Ideal of vanishing polynomials]}:
Given an ideal $J\subset R$ and $V(J) \subseteq \Fq^d$, the {\it ideal
of polynomials that vanish on} $V(J)$ is $I(V(J)) = \{ f \in R :
\forall \bm{a} \in V(J), f(\bm{a}) = 0\}$.
\end{Definition}

If $f$ vanishes on $V(J)$, then $f \in I(V(J))$. 

\begin{Definition}{\it [Radical ideal]}:
For any ideal $J\subset R$, the {\it radical} of $J$ is defined
as $\sqrt{J} = \{f \in R: \exists m \in \mathbb{N} ~s.t. f^m \in J\}.$
\end{Definition}

When $J = \sqrt{J}$, then $J$ is called a {\it radical ideal}.
Over algebraically closed fields, the {\it Strong Nullstellensatz}
establishes the correspondence between radical ideals and varieties. 

\begin{Definition}{\it [Maximal ideal]}: An ideal $J\subset R$ is 
said to be {\it maximal} if $J \neq R$ and any
ideal $J'$ containing $J$ is such that either $J'=J$ or $J'=R$.
\end{Definition}

\begin{Lemma}
\label{lemma:radical-ff}
(From \cite{gao:qe-gf-gb}) For an arbitrary ideal $J\subset
\Fq[x_1,\dots,x_d]$, and  $J_0 = \langle
x_1^q-x_1,\dots,x_d^q  -  x_d\rangle$, the ideal $J + J_0$ is radical;  
i.e. $\sqrt{J+J_0} = J+J_0$. 
\end{Lemma}

The Strong Nullstellensatz, which has a special form over finite fields,
characterizes the ideal $I(V(J))$.

\begin{Theorem}{\it [The Strong Nullstellensatz over finite fields
(Theorem 3.2 in \cite{gao:qe-gf-gb})]}: \label{thm:strong-ns}  
For any ideal $J \subset \Fq[x_1,\dots,x_d]$$, ~I(V(J)) = J + J_0$.
\end{Theorem}

% {\it Vanishing Polynomials:} For any element $\varphi\in\Fq$, $\varphi^q=\varphi$ holds. Therefore,
% the polynomial $x^q-x$ vanishes everywhere in $\Fq$, and we call it a
% {\it vanishing polynomial}. We denote by $F_0 =
% \{x_1^q-x_1,\dots,x_d^q-x_d\}$ the set of all vanishing polynomials in
% $R$, and  $J_0 = \langle F_0 \rangle$ denotes the ideal of all
% vanishing polynomials in $R$. Then $V_{\Fq}(J_0) = \Fq^d$, and for any
% ideal $J$, we have that $V_{\Fq}(J)=V_{\Fq}(J+J_0)$.

% {\it Ideal-Variety Correspondences:} Given two ideals $J_1 = \langle
% f_1,\dots,f_s\rangle,$ $J_2=\langle h_1,\dots,h_r\rangle$, the {\bf
%   sum of ideals} is denoted as $J_1 + J_2 = \langle
% f_1,\dots,$ $f_s,h_1,\dots,h_r\rangle$, their {\bf product} is
% given as $J_1\cdot J_2 = \langle f_i\cdot h_j: 1\leq i\leq s, 1\leq
% j\leq r\rangle$, and a {\bf colon ideal} operation is defined as
% $J_1:J_2 = \{f \in R\ |\ f\cdot g \in J_1, \forall g \in J_2\}$. 
% Ideals and varieties are dual concepts:
% $V(J_1 + J_2) = V(J_1) \cap V(J_2)$, whereas $V(J_1\cdot J_2) = V(J_1)
% \cup V(J_2)$. A colon ideal operation corresponds the set difference of 
% two varieties, $V_{\Fq}(J_1:J_2)=V_{\Fq}(J_1)
% \setminus V_{\Fq}(J_2)$. 
% % , where ``$\setminus $'' denotes the set difference. 
% The complement of a variety $V_{\Fq}(J_1)$ denoted $\overline{V_{\Fq}(J_1)}$, 
% can be computed using the colon ideal with vanishing polynomials 
% as: $\overline{V_{\Fq}(J_1)} = \Fq^d \setminus
% V_{\Fq}(J_1) = V_{\Fq}(J_0) \setminus V_{\Fq}(J_1) = V_{\Fq}(J_0:J_1)$.
% A \Grobner basis is essentially a canonical representation of an
% ideal. 
% The operations of sum, product, and colon ideals --
% $J_1+J_1,J_1\cdot J_2,$ and $J_1:J_2$, respectively, are all performed 
% using their \Grobner bases. 

% {\it As varieties over finite fields are a finite set of points in $\Fq^d$,
% in this work we interpret the desired rectification functions as
% varieties. Thus, we use the above concepts of intersection, union, and
% complement of varieties to operate on functions; however, we compute
% them algebraically using the \Grobner bases of corresponding ideals.}

% GB computations in $\F_q$ exhibit exponential complexity. 
% In this term order, the variables (nets of the circuit) are ordered
% based on a reverse topological traversal of the circuit from POs to
% PIs. This specific term order '$>$` is called the {\it Reverse
%   Topological Term Order (RTTO)}. Our MFR approach also uses RTTO $>$
% for polynomial representation and manipulation. 


\bif{Reverse Topological Term Order:}
The computational complexity of the GB algorithm is exponential
in the number of variables $d$.
For efficient GB computations on polynomials derived from a circuit,
\cite{lv:tcad2013} proposed the use of a specialized term order that
exploits the topology of the circuit and renders the set of polynomials $F=\{f_1,\dots,f_s\}$ for 
the gates of the circuit, a \Grobner basis itself. 
This term order '$>$`
is called the {\it Reverse Topological Term Order (RTTO)}.

\begin{Definition}\label{def:rtto}
{\it  [Reverse Topological Term Order]}~\cite{lv:tcad2013}:
Let $C$ be an arbitrary combinational
circuit. Let $\{x_1, \dots$ $, x_d\}$ denote the set of all variables
(signals) in $C$. Starting from the primary outputs, perform
a {\it reverse topological traversal} of the circuit and order the
variables such that $x_k > x_j$ if $x_k$ appears earlier in the
reverse topological order. Impose a lex term order $>$ to represent each
gate as a polynomial $f_j$, s.t. $f_j = x_k + tail(f_j)$. 
As each gate output is a distinct net, the
leading terms of all polynomials in $F$ become relatively prime.
Then, the set of polynomials $F=\{f_1,\dots,f_s\}$ corresponding to the gates of the circuit 
is a \Grobner basis when RTTO is used for ordering.
\end{Definition}

Consequently, remainders 
  generated by polynomial division using RTTO $>$ comprise only primary input variables.
  Our rectification approach also uses RTTO $>$
for polynomial representation and manipulation.

%To overcome the complexity of GB computations,
% While the computations are based on GB theory, 
%we use a specialized term order~\cite{lv:tcad2013} 
%to exploit the topology of the circuit.
%Our verification and rectification tests follow from the Strong Nullstellensatz
%%%%% RTTO

%% \begin{Definition}[Elimination Ideal]
%% Given an ideal $J = \langle f_1, \dots, f_s\rangle \subset
%% \Fq[x_1,\dots,x_d]$, the $l$-th elimination ideal $J_l$ is defined as
%% $J_l = J \cap \Fq[x_{l+1},\dots,x_d]$. 
%% \end{Definition}
%% The ideal $J_l$ is called an elimination ideal because it eliminates
%% variables $x_1,\dots,x_{l}$. Generators of the $l$-th
%% elimination ideal can be computed using \Grobner bases using {\it lex}
%% term orders.

%% \begin{Theorem}[Elimination Theorem]
%% \label{def:elim}
%% Let $J \subset R$ be an ideal and $G$ be its \Grobner basis w.r.t. the
%% lexicographical (lex) order on the variables where $x_1 > x_2 > \cdots
%% > x_d$. Then for every $0 \leq l \leq d$ the set $G_l = G \cap
%% \Fq[x_{l+1},\dots,x_d]$ is a \Grobner basis of the $l^{th}$
%% elimination ideal $J_l$. 
%% \end{Theorem}


%using elimination ideals.  
%
%% % \par {\bf Weak Nullstellensatz and Elimination Theory:} 
%% \begin{Theorem}[{\it The Weak Nullstellensatz over finite fields (from
%% Theorem 3.3 in~\cite{gao:gf-gb-ms})}]
%% \label{thm:weak-ns-ff}
%% {\it For a finite field $\Fq$ and the ring $R = \Fq[x_1, \dots, x_d]$, let
%% $J = \langle f_1, \dots, f_s\rangle \subseteq R$, and let $J_0 = \langle
%% x_1^q-x_1, \dots, x_d^q -  x_d\rangle$ be the ideal of vanishing
%% polynomials. Then $V_{\Fq}(J) = \emptyset \iff 1 \in J + J_0 \iff G =
%% GB(J+J_0) = \{1\}$. }

%% \par To find whether a set of polynomials $f_1,\dots,f_s$ have no common
%% zeros in $\Fq$, we can compute the GB $G$ of
%% $\{f_1,\dots,f_s,x_1^q-x_1,\dots,x_d^q-x_d\}$ and see if $G = \{1\}$. 
%% \end{Theorem}


%% We also need to employ notion of difference of varieties in our theoretical
%% section. The equivalent ideal operation is called the quotient of ideals.
%% In terms of varieties, $V_{\Fq}(J_1:J_2) = V_{\Fq}(J_1) \setminus V_{\Fq}(J_2)$.
%% \par The computer algebra tools like SINGULAR~\cite{DGPS_410} contain implementations for 
%% computing elimination ideals and quotient of ideals.
%% \subsection{Circuit Polynomials}
%% \label{composite_field}
%% For a given data-path size $n$, $q=2^n$ is
%% chosen to
% takes W which takes values in $\Fkm$.
% change f to fspec.
% f1 to fs are gates of the circuit. 
% \vspace{-0.10in}
