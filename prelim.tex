\section{Preliminaries}
\label{sec:prelim}
% This section reviews some basic concepts from symbolic computer
% algebra and associated algorithms that are utilized in this paper. 
%\subsection{Notation and Background}
{\it Finite fields:} Let $\B$ denote the Boolean domain, and $\neg,
\wedge,\vee, \oplus$ the NOT, AND, OR and XOR operators,
respectively. Let $\F_2=\{0,1\}$ be the field of 2 elements, and let
$\F_q = \F_{2^n}$ denote the finite field of $q=2^n$ elements, for a
given $n\in \Z_{> 0}$. $\Fn$ is the $n$-dimensional extension of
$\F_2$, and it is constructed as $\Fn = \F_2[x]\pmod{ P_n(x)}$. Here
$P_n(x) \in \F_2[x]$ is a given degree-$n$ polynomial, {\it
  irreducible} in $\F_2$, with a root $\gamma$, i.e.  $P_n(\gamma)=0$.  
%, which is a generator of
% $\Fn$. Then $\gamma$ is called a primitive element (PE) of $\Fn$ and
% it generates the entire field: $\Fn=\{0,1=\gamma^{2^n - 1}, \gamma,
% \gamma^2,\dots,\gamma^{2^n - 2}\}$. Note: $\forall \gamma\in \Fkn, \gamma^{2^n-1}=1$.
% An element $A \in \Fn$ can be written as $A = a_0 + a_1\cdot \gamma +
% \dots + a_{n-1}\cdot\gamma^{n-1}$, where $a_0,\dots,a_{n-1} \in
% \F_2$. 
%% In $\Fq$, the addition ($''+''$) and multiplication
%% ($''\cdot''$) operations are performed in the base field $\F_2$ and
%% reduced modulo the corresponding primitive polynomial $P_n(x)$. For
%% $n, k \in \Z_{> 0}$, if $n$ divides $k$, then $\Fn \subset
%% \Fkk$.
Thus, $\F_2 \subset \F_{2^n}, \forall n>1$. All fields of the type
$\Fkn$ have characteristic 2, and therefore $-1 = +1$ in $\Fkn$. 


{\it Polynomial representation:} Let $R=\F_q[x_1,\dots,x_d]$ be the
polynomial ring in variables $x_1,\dots,x_d$ with coefficients in
$\F_q$. A polynomial $f \in R$ is 
written as a finite sum of terms  $f = c_1 M_1 +  c_2 M_2 + \dots +
c_p M_p$, where $c_1, \dots, c_p$ are coefficients from $\F_q$ and
$M_1, \dots, M_p$ are monomials, i.e. power products of the type
$x_1^{e_{1}}\cdot x_2^{e_{2}}\cdots x_d^{e_{d}}$,  $e_j \in \Z_{\geq  0}$. 
% To systematically manipulate the polynomials, a monomial order $>$ (or
% a term order) is imposed on $R$ such that the monomials of all
% polynomials are ordered according to $>$. 
A monomial order $>$ (term order) is always imposed on $R$, and
subject to $>$, we have that $M_1 >M_2 > \dots > M_p$. Then, $lm(f) =
M_1$ denotes the {\it leading monomial} of $f$, and $lt(f) = c_1 M_1$
the  {\it leading   term} of $f$. 
%Also, for $f\in R$, $tail(f) = f - lt(f)$.
%This work employs {\it lexicographic} (lex) term orders.  

%%%%%%%%%%%%%%%%%%%% PK
%% A circuit with $n$-bit operands is modeled as a polynomial function in
%% $\Fn$, where variables $x_1,\dots,x_d$ denote the nets of the circuit.
%% %%%% Move to verification
%% Logic gates of the circuit can be modeled with polynomials in
%% $\F_2 \subset \Fn$, with the mapping $\B \mapsto \F_2$:
%% %% As $\Fkk \supset \F_2$, these polynomials can also be
%% %% construed as polynomials in $\Fkk$.
%% %% given as: 
%% \begin{equation}
%% \label{bool2poly}
%% \begin{split}
%% z ~ =  ~ \neg a ~ \rightarrow ~ z+a+1 & \pmod 2  \\
%% z ~ =  ~ a \wedge b ~ \rightarrow ~ z+a \cdot b & \pmod 2\\
%% z ~ =  ~ a \vee b ~ \rightarrow ~ z+a+b+a \cdot b & \pmod 2 \\
%% z ~ =  ~ a \oplus b ~ \rightarrow ~ z+a+b & \pmod 2 
%% \end{split}
%% \end{equation}

{\it Polynomial Reduction:} Let $F=\{f_1,\dots,f_s\}$ be
a set of polynomials in $R$ and $f\in R$ be 
another polynomial. Then $f\xrightarrow{F}_+r$ denotes the {\it
  reduction} of $f$ modulo the set of polynomials $F$ resulting in a
remainder $r$, obtained by iteratively canceling terms in $f$ by
$lt(f_j), f_j\in F$, via polynomial division (cf. Algorithm
1.5.1~\cite{gb_book}). 
%The
%remainder $r$ is said to be {\it reduced} such that no term in $r$ is
%divisible by the leading term of any $f_j \in F$.

%Algorithm~\ref{algo:mv_reduce}
% term of any polynomial $f_j$ in $F$.
% Along with the remainder $r$, the algorithm also returns
% the set of quotients $\{u_1,\dots,u_s\}$ of division of $f$ by
% $\{f_1,\dots,f_s\}$, respectively, such that $f = u_1\cdot
% f_1+\dots+u_s\cdot f_s + r$.}

% {\small
% \begin{algorithm}[hbt]
%  \caption{Multivariate Reduction of $f$ by $F=\{f_1,\dots,f_s\}$}
%  \label{algo:mv_reduce}
%  \begin{algorithmic}[1]
%  % \Procedure{$multi\_variate\_division$}{$f, f_1, \dots, f_s \in \F[x_1, \dots, x_d], f_i\neq 0$}
%  \Procedure{$multi\_var\_division$}{$f,\{f_1,\dots,f_s\},f_j\neq0$}
%  % \ENSURE $u_1,\dots, u_s, r$ s.t. $f = \sum f_i u_i+r$ where $r$ is
%  % reduced w.r.t. $F = \{f_1,\dots, f_s\}$ and max($lp(u_1)lp(f_1), \dots, lp(u_s)lp(f_s), lp(r)$) = $lp(f)$
%  \State $u_j \gets 0; ~r \gets 0, ~h \gets f $ 
%  \While {  $h \neq 0$ }
%  \If{ $\exists j$ s.t. $lm(f_j) ~|~ lm(h)$}
%  \State choose $j$ least s.t. $lm(f_j) ~|~ lm(h)$
%  \State $u_j = u_j + \frac{lt(h)}{lt(f_j)}$
%  \State $h = h - \frac{lt(h)}{lt(f_j)} f_j$
%  \Else
%  \State $r = r+ lt(h)$
%  \State $h = h - lt(h)$
%  \EndIf
%  \EndWhile
%  \State \Return $(\{u_1,\dots,u_s\} , r)$
%  \EndProcedure
%  \end{algorithmic}
%  \end{algorithm}
% }

%% The algorithm initializes $h$ with the polynomial $f$ and cancels its
%% leading term by some  polynomial $f_j$. If the leading term $lt(h)$
%% cannot be canceled by any $lt(f_j)$, then it is added to the  final
%% remainder $r$ and the process is repeated until all the terms in $h$
%% are analyzed.  
%% We represent the given circuit by way of a set of polynomials
%% $F=\{f_1,\dots,f_s\}$, and for verification and rectification, we
%% analyze the solutions to the polynomial equations $f_1=\dots=f_s=0$. 
%% For this purpose, we
%% consider the {\it ideal} generated by the polynomials, and their {\it
%%   variety.} 

%\begin{Definition}$\bf{\left[Ideal\right]}$
{\it Ideals and varieties:} A given set of polynomials 
$F=\{f_1,\dots,f_s\}$ from $R$, generates the {\bf ideal} $J = \langle
F \rangle \subseteq R$, defined as $J = \langle f_1,\dots,$ $ f_s \rangle = \{
h_1\cdot f_1 + \dots+h_s\cdot f_s~|~h_1,\dots,h_s\in R\}$. 
Polynomials $f_1,\dots,f_s$ form the generators or the {\it basis} of ideal $J$.


% {\it Ideals and Varieties:} A set of polynomials 
% $F=\{f_1,\dots,f_s\}$ from $R$, generates the {\bf ideal} $J = \langle
% F \rangle \subseteq R$, defined as $J = \langle f_1, \dots, f_s \rangle = \{
% h_1\cdot f_1 + \dots+h_s\cdot f_s:  h_1,\dots,h_s\in R\}$. 
% %% \vspace{-0.1in}
% %% \begin{small}
% %% \begin{equation}
% %% J = \langle f_1, \dots, f_s \rangle = \{ h_1\cdot f_1 + \dots+h_s\cdot
% %% f_s:  h_1,\dots,h_s\in R\}
% %% \end{equation}
% %% \end{small}
% %% \vspace{-0.1in}
% Polynomials $f_1,\dots,f_s$ form the basis or generators of ideal $J$.
%\end{Definition}

Let $\bm{a} = (a_1,\dots,a_d) \in \Fq^d$ be a point in the affine
space, and $f$ a polynomial in $R$. If $f(\bm{a}) = 0$, we say
that $f$ {\it vanishes} on $\bm{a}$. In this work, we have to analyze
the {\it set of all common zeros} of the polynomials of $F = 
\{f_1,\dots,f_s\}$  that lie within the field $\Fq$ -- i.e. the set of
all point $\bm{a} \in \Fq^d$ such that
$f_1(\bm{a})=\dots=f_s(\bm{a})=0$. This zero set is called the {\bf
  variety}, which depends not just on the given set of polynomials in
$F$, but rather on the ideal generated by polynomials. We denote it by
$V(J)$, where: $ V(J)= V_{\Fq}(J) = V_{\Fq}(f_1,\dots,f_s) = \{\bm{a} 
\in \Fq^d: \forall f \in J, f(\bm{a}) = 0\}.$
% {\it \Grobner Bases:} An ideal $J$ may have many different sets of 
% generators. %such that their varieties are the same.
% A \Grobner basis (GB) is one such generating set $G=\{g_1,\dots,g_t\}$
% with special properties that helps to solve many polynomial decision
% and quantification problems.  
 % that is a canonical representation of the ideal. 

An ideal may have many different bases. A {\it \Grobner basis} (GB) is
a basis with special properties that allows to solve many polynomial
decision and quantification problems. 
\begin{Definition}
\label{def:gb}
{\bf [\Grobner Basis]}~\cite{gb_book}: 
For a monomial ordering $>$, a set of non-zero polynomials $G =
\{g_1,g_2,\cdots,g_t\}$ contained in an ideal $J$, is called a
\Grobner Basis (GB) of $J$ $\iff$
$\forall f \in J$, $f\xrightarrow{g_1,\dots,g_t}_+0$. 
%$i \in \{1,\cdots, t\}$ 
\end{Definition}
% The GB $G$ for an ideal $J$ can be computed using the  Buchberger's
% algorithm  %\cite{buchberger_thesis}. 
% (cf. Alg. 1.7.1 in~\cite{gb_book}), which
% takes as input a set of polynomials $F = \{f_1,\dots, f_s\}$ and
% computes its GB $G = \{g_1,\dots,g_t\}$, such that $J = \langle
% F\rangle = \langle G\rangle$. Moreover, $V(J) = V(F) = V(G)$. 
% The GB $G$ for an ideal $J$ can be computed using the  Buchberger's
% algorithm  %\cite{buchberger_thesis}. 
% (cf. Alg. 1.7.1 in~\cite{gb_book})
A polynomial $f$ is a member of ideal $J$ iff division of $f$ by
$GB(J)$ gives the remainder 0. If $f \notin J$,
then division by $GB(J)$, $f\xrightarrow{GB(J)}_+r$, results in a
non-zero remainder $r$ that is unique.
A \Grobner basis can be computed using Buchberger's 
algorithm (cf. Alg. 1.7.1 in~\cite{gb_book}). 
A \Grobner basis can be further reduced. A reduced \Grobner basis
(redGB) is canonical representation of the ideal $J$.

\begin{Definition} \label{def:rgb}
{\bf [Reduced \Grobner Basis]}:
    A redGB for a polynomial ideal $J$ is 
    a GB $G=\{g_{1},\dots,g_{t}\}$ such that:
    \begin{itemize}
        \item $lc(g_{i})=1,\forall g_{i}\in G$
        \item $\forall g_{i} \in G$, no monomial of $g_{i}$ 
        lies in $\langle lt(G-\{g_{i}\})\rangle$
    \end{itemize}
\end{Definition}

%% {\it Ideal-Poly Conversion:} Over finite fields, given an ideal $J$, 
%% a polynomial $p$ can always be computed 
%% such that $V(p) = V(J)$. Let $\{g_1,\dots,g_t\}$ denote the generators
%% of $J$. Then, the polynomial $p$ can be constructed as, 
%% $p = (1+g_1)(1+g_2)\dots(1+g_t)+1$~\cite{Utkarsh:VLSI18}. 

% A reduced \Grobner basis
% is computed by first computing a minimal \Grobner basis, and then reducing it.

{\it Vanishing Polynomials:} For any element $\varphi\in\Fq$, $\varphi^q=\varphi$ holds. Therefore,
the polynomial $x^q-x$ vanishes everywhere in $\Fq$, and we call it a
{\it vanishing polynomial}. We denote by $F_0 =
\{x_1^q-x_1,\dots,x_d^q-x_d\}$ the set of all vanishing polynomials in
$R$, and  $J_0 = \langle F_0 \rangle$ denotes the ideal of all
vanishing polynomials in $R$. Then $V_{\Fq}(J_0) = \Fq^d$, and for any
ideal $J$, we have that $V_{\Fq}(J)=V_{\Fq}(J+J_0)$.

{\it Ideal-Variety Correspondences:} Given two ideals $J_1 = \langle
f_1,\dots,f_s\rangle,$ $J_2=\langle h_1,\dots,h_r\rangle$, the {\bf
  sum of ideals} is denoted as $J_1 + J_2 = \langle
f_1,\dots,$ $f_s,h_1,\dots,h_r\rangle$, their {\bf product} is
given as $J_1\cdot J_2 = \langle f_i\cdot h_j: 1\leq i\leq s, 1\leq
j\leq r\rangle$, and a {\bf colon ideal} operation is defined as
$J_1:J_2 = \{f \in R\ |\ f\cdot g \in J_1, \forall g \in J_2\}$. 
Ideals and varieties are dual concepts:
$V(J_1 + J_2) = V(J_1) \cap V(J_2)$, whereas $V(J_1\cdot J_2) = V(J_1)
\cup V(J_2)$. A colon ideal operation corresponds the set difference of 
two varieties, $V_{\Fq}(J_1:J_2)=V_{\Fq}(J_1)
\setminus V_{\Fq}(J_2)$. 
% , where ``$\setminus $'' denotes the set difference. 
The complement of a variety $V_{\Fq}(J_1)$ denoted $\overline{V_{\Fq}(J_1)}$, 
can be computed using the colon ideal with vanishing polynomials 
as: $\overline{V_{\Fq}(J_1)} = \Fq^d \setminus
V_{\Fq}(J_1) = V_{\Fq}(J_0) \setminus V_{\Fq}(J_1) = V_{\Fq}(J_0:J_1)$.
% A \Grobner basis is essentially a canonical representation of an
% ideal. 
% The operations of sum, product, and colon ideals --
% $J_1+J_1,J_1\cdot J_2,$ and $J_1:J_2$, respectively, are all performed 
% using their \Grobner bases. 

{\it As varieties over finite fields are a finite set of points in $\Fq^d$,
in this work we interpret the desired rectification functions as
varieties. Thus, we use the above concepts of intersection, union, and
complement of varieties to operate on functions; however, we compute
them algebraically using the \Grobner bases of corresponding ideals.}

GB computations in $\F_q$ exhibit exponential complexity. For
efficient GB computations on polynomials derived from a circuit,
\cite{lv:tcad2013} proposed the use of a specialized term order that
exploits the topology of the circuit. 
In this term order, the variables (nets of the circuit) are ordered
based on a reverse topological traversal of the circuit from POs to
PIs. This specific term order '$>$` is called the {\it Reverse
  Topological Term Order (RTTO)}. Our MFR approach also uses RTTO $>$
for polynomial representation and manipulation. 


%To overcome the complexity of GB computations,
% While the computations are based on GB theory, 
%we use a specialized term order~\cite{lv:tcad2013} 
%to exploit the topology of the circuit.

%% \begin{Definition}[Elimination Ideal]
%% Given an ideal $J = \langle f_1, \dots, f_s\rangle \subset
%% \Fq[x_1,\dots,x_d]$, the $l$-th elimination ideal $J_l$ is defined as
%% $J_l = J \cap \Fq[x_{l+1},\dots,x_d]$. 
%% \end{Definition}
%% The ideal $J_l$ is called an elimination ideal because it eliminates
%% variables $x_1,\dots,x_{l}$. Generators of the $l$-th
%% elimination ideal can be computed using \Grobner bases using {\it lex}
%% term orders.

%% \begin{Theorem}[Elimination Theorem]
%% \label{def:elim}
%% Let $J \subset R$ be an ideal and $G$ be its \Grobner basis w.r.t. the
%% lexicographical (lex) order on the variables where $x_1 > x_2 > \cdots
%% > x_d$. Then for every $0 \leq l \leq d$ the set $G_l = G \cap
%% \Fq[x_{l+1},\dots,x_d]$ is a \Grobner basis of the $l^{th}$
%% elimination ideal $J_l$. 
%% \end{Theorem}


%using elimination ideals.  

%Our verification and rectification tests follow from the Strong Nullstellensatz
%%%%% RTTO
%% \begin{Definition}
%% \label{def:rtto}
%% \par {\it Reverse Topological Term Order~\cite{lv:tcad2013}:}
%% The computational complexity of Buchberger's algorithm is exponential
%% in the number of variables $n$. As our work is focused on the circuits,
%% we will describe a term order that renders the set of polynomials for 
%% the gates of the circuit, a \Grobner basis itself. This term order 
%% is called Reverse Topological Term Order (RTTO).

%% \par Let $C$ be an arbitrary combinational
%% circuit. Let $\{x_1, \dots$ $, x_d\}$ denote the set of all variables
%% (signals) in $C$. Starting from the primary outputs, perform
%% a {\it reverse topological traversal} of the circuit and order the
%% variables such that $x_k > x_j$ if $x_k$ appears earlier in the
%% reverse topological order. Impose a lex term order $>$ to represent each
%% gate as a polynomial $f_j$, s.t. $f_j = x_k + tail(f_j)$. Then 
%% set of polynomials $\{f_1,\dots,f_s\}$ corresponding to the gates of the circuits 
%% is a \Grobner basis when RTTO is used for ordering.
%% \end{Definition}
%% % \par {\bf Weak Nullstellensatz and Elimination Theory:} 
%% \begin{Theorem}[{\it The Weak Nullstellensatz over finite fields (from
%% Theorem 3.3 in~\cite{gao:gf-gb-ms})}]
%% \label{thm:weak-ns-ff}
%% {\it For a finite field $\Fq$ and the ring $R = \Fq[x_1, \dots, x_d]$, let
%% $J = \langle f_1, \dots, f_s\rangle \subseteq R$, and let $J_0 = \langle
%% x_1^q-x_1, \dots, x_d^q -  x_d\rangle$ be the ideal of vanishing
%% polynomials. Then $V_{\Fq}(J) = \emptyset \iff 1 \in J + J_0 \iff G =
%% GB(J+J_0) = \{1\}$. }

%% \par To find whether a set of polynomials $f_1,\dots,f_s$ have no common
%% zeros in $\Fq$, we can compute the GB $G$ of
%% $\{f_1,\dots,f_s,x_1^q-x_1,\dots,x_d^q-x_d\}$ and see if $G = \{1\}$. 
%% \end{Theorem}


%% We also need to employ notion of difference of varieties in our theoretical
%% section. The equivalent ideal operation is called the quotient of ideals.
%% In terms of varieties, $V_{\Fq}(J_1:J_2) = V_{\Fq}(J_1) \setminus V_{\Fq}(J_2)$.
%% \par The computer algebra tools like SINGULAR~\cite{DGPS_410} contain implementations for 
%% computing elimination ideals and quotient of ideals.
%% \subsection{Circuit Polynomials}
%% \label{composite_field}
%% For a given data-path size $n$, $q=2^n$ is
%% chosen to
% takes W which takes values in $\Fkm$.
% change f to fspec.
% f1 to fs are gates of the circuit. 
% \vspace{-0.10in}
