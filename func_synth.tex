\section{Synthesizing Rectification Functions}\label{comp:synth}

The above techniques show how to construct a rectification function
by reasoning about the varieties of remainders $rem_l$, $1\leq l \leq 2^m$.  
% The techniques described above evaluate an unknown rectification
% function $u_i$ by reasoning about its variety and the variety of the remainders $rem_l$.
However, algebraically, we compute these functions using their corresponding ideals.
In this section, we show how the ideals corresponding to the remainders computed in 
Theorem~\ref{Thm:rect} can be utilized for rectification function computation.
% The rectification theorem~\ref{thm:rectcheck} implies the {\it existence of a
% polynomial function} with mapping $u_i:\F_2^{|X_{PI}|}\rightarrow \F_2$, at individual
% targets $w_i$. 
% Since we use a polynomial model over $\Fn$, our algebraic rectification techniques 
% may compute a polynomial function over $\Fn$, i.e., it may evaluate to non-Boolean (non-synthesizable). 
% For this purpose, we restate Theorem 8.5 and its proof from~\cite{UTK:thesis}, which we utilize. 

% Since it is not practical to compute varieties, we provide a procedure to
% obtain the rectification functions from the generators of these remainders. 

% \begin{Theorem}
% \label{thm:coffF2}
% Given a polynomial ring $\Fn[x_1,\dots,x_d]$ and a variety $V \subseteq \mathbb{F}_2^d$,
% the ideal $\mathcal{I}_{\Fn}(V)$ that vanishes on $V$ and has coefficients in $\Fn$ 
% can be expressed as the product $A \cdot\Fn[x_1,\ldots,x_d]$, where $A = \bigcap_{a\in V}m_a$
% and $m_a = \langle x_1-a_1,\dots,x_d-a_d \rangle \subseteq \F_2[x_1, \ldots, x_d]$ for 
% each $a=(a_1,\dots,a_d) \in V$.
% \end{Theorem}

% \begin{proof}
% Consider a single point $a \in V$ and the ideal $\mathcal{I}_{\Fn}(\{a\})$ which has coefficients in 
% $\Fn$ and vanishes on $a$. This ideal can be written as,
% \begin{align*}
% \mathcal{I}_{\Fn}(\{a\})=\mathcal{I}_{\overline{\Fn}}(\{a\})\cap\Fn[x_1,\dots,x_d]
% \end{align*}
% The above expression can further simplified as follows,
% \begin{align*}
% \mathcal{I}_{\overline{\Fn}}(\{a\})\cap \Fn[x_1,\ldots,x_d] 
% &= (m_a\cdot \overline{\Fn}[x_1,\ldots,x_d])\cap\Fn[x_1,\ldots,x_d] \\
% &= m_a\cdot \Fn[x_1,\ldots,x_d]
% \end{align*}

% Now consider the ideal $\mathcal{I}_{\Fn}(V)$ which can written as the intersection of 
% $\mathcal{I}_{\Fn}(\{a\})$ for each $a \in V$,
% \begin{align*}
% \mathcal{I}_{\Fn}(V) &= \bigcap_{a\in V}\mathcal{I}_{\Fn}(\{a\}) \\
% &= \bigcap_{a\in V}m_a\cdot \Fn[x_1,\ldots,x_d] \\
% &= (\bigcap_{a\in V}m_a)\cdot \Fn[x_1,\ldots,x_d],
% \end{align*}
% where the last equality is due to $\F_2[x_1,\ldots,x_d] \subseteq \Fn[x_1,\ldots,x_d]$.
% \end{proof}

Consider the remainders $rem_l$ ($\langle rem_l \rangle \subseteq \Fn[X_{PI}] $) computed in Theorem~\ref{Thm:rect}. 
Since we use a polynomial model over $\Fn$, the remainders $rem_l$ computed using our algebraic rectification techniques 
may evaluate to non-Boolean values (non-synthesizable). In other words, each $rem_l$ is a 
polynomial function with mapping of the form $f:\F_2^{|X_{PI}|}\mapsto \Fn$. However, we are interested
in operating on synthesizable functions in order to derive a patch sub-circuit.
To achieve this, we present techniques to 
transform a polynomial function of the form $rem_l$ to another polynomial $p_{G_l}$ 
such that $V(\langle rem_l \rangle) = V(\langle p_{G_l} \rangle)$, and $p_{G_l}$ 
% evaluates only in $\F_2$.
is a function with mapping of the form $f:\F_2^{|X_{PI}|}\mapsto \F_2$.

Consider an ideal $J\subseteq R = \Fq[x_1,\dots,x_d]$ and let $J_0 = \{x_i^q-x_i: 1\leq i \leq d\}$. 
Given that ideals of the form $J + J_0$ contain a finite variety, 
and that they are radical, results from algebraic geometry show: 

\begin{Fact}
The ideal $ J+ J_0$ can be expressed as the intersection of maximal ideals. For radicals, the intersection is the same as their product: $J+J_0 = \bigcap_{i=1}^{k}M_i = \prod_{i=1}^{k}M_i$, where $M_i$ are maximal in $R$. Furthermore, each maximal ideal $M_i$ is of the form $M = (x_1-\epsilon_1,\dots,x_d-\epsilon_d)$, where $\epsilon_j \in \F_2, \forall 1\leq j \leq d$.
\end{Fact}
% Every ideal can be written as an intersection of prime ideals. In a Finite Set/Field: irreducible prime ideal is only one point. J + J_0 is a finite set of points. Can decompose to union of individual points. ideal which corresponds to individual point is a maximal ideal of form x - a, by defn is prime. Every ideal of a finite variety can be decomposed into an intersection of maximal ideals. finite ideal - irreducible = 1 pt. Prop 6.1 = fact.  

% \begin{Proposition}
% The ideal $J+J_0\pmod{J_0}$ can be expressed as intersection of maximal ideals, 
% i.e., $J+J_0\pmod{J_0} = \bigcap_{i=1}^{k}M_i = \prod_{i=1}^{k}M_i$, where $M_i$ are maximal in $R$.
% \label{propA}
% \end{Proposition}

% % \begin{Proposition}
% % Intersection of maximal ideals can be computed as products, i.e., $\bigcap_{i=1}^{k}M_i = \prod_{i=1}^{k}M_i$.
% % \label{propB}
% % \end{Proposition}

% % Also included in fact from before.

% \begin{Proposition}
% Each maximal ideal $M$ of $R$ is of the form $M = (x_1-\epsilon_1\cdot1,\dots,x_d-\epsilon_d\cdot1)$,
% where $\epsilon_j \in \{0,1\}, \forall 1\leq j \leq d$.
% \label{propB}
% \end{Proposition}

% Write an example for this proposition. Define product of ideals. Kalla will give example.  Significance: When we compute remainder rem_l, GB(rem_l, J0) will have polys with coef only in {+-1}, as in example sent to prof. Florian

\begin{Proposition}
The ideal $\prod_{i=1}^{k}M_i$ can be expressed as generators $(f_1,\dots,f_s,J_0)$ such that
$f_1,\dots,f_s$ have coefficients only in $\F_2$.
\end{Proposition}

The above observations imply that any arbitrary polynomial $f \in R$, when combined with $J_0$, can be expressed 
as $\langle f\rangle+ J_0  = \langle g_1, \dots, g_t, J_0 \rangle$, where $g_1, \dots, g_t$ have coefficients 
in $\F_2$.
It is a property of such ideals ($\langle f \rangle + J_0 $) that the generators of their reduced
\Grobner bases (Def.~\ref{def:rgb}) have coefficients only in $\Ftwo$. 

\begin{Proposition}\label{prop:singpoly}
Given a set of generators $G = \{g_1,\dots,g_t\}$ with coefficients only in $\F_2$, a polynomial
$p$ can always be constructed as $p = ((1+g_1)(1+g_2)\dots(1+g_t)+1)$, such that 
$V(\langle p\rangle) = V(\langle g_1,\dots,g_t\rangle+J_0)$ and $p$ evaluates 
to values in $\F_2$.
\end{Proposition}
It is easy to assert that $V(\langle p \rangle) = V(G)$. 
Consider a point $\bm{a} \in V(G)$. As all of 
$g_1,\dots,g_t$ vanish (= 0) at $\bm{a}$,
\begin{align*}
p(\bm{a}) &= (1+g_1(\bm{a}))(1+g_2(\bm{a}))\cdots(1+g_t(\bm{a}))+1 \\
&= (1+0)(1+0)\cdots(1+0) + 1 = 0
\end{align*}
Conversely, for a point $\bm{a'} \not \in V(G)$, at least one  
of $g_1,\dots,g_t$ will evaluate to 1. 
Without loss of generality, if
$g_1$ evaluates to 1 at $\bm{a'}$, then $p=(1+1)(1+0)\cdots(1+0)+1 = 1 \neq
0$. In addition to $V(\langle p\rangle) = V(G)$, $p$ either evaluates to 0 or 1. 
%%%%%%%%% The algorithm is commented out
%% \par Using Eqn. (\ref{eqn:id2poly}), a recursive procedure is derived to
%% compute $U$, and it is depicted in Algorithm \ref{algo:id2poly}. At
%% every recursive step, we also reduce the intermediate  
%% results by $\pmod{ J_0}$ (line 7) so as to avoid terms of high degree.

% In this fashion, from the ideal-interpolant $E_L$, we compute the
% single-fix rectification polynomial function $\uc$, and synthesize a
% sub-circuit at net $x_i$ such that $x_i = \uc$ rectifies the circuit. 

Even though the remainders $rem_l$ have coefficients in $\Fkn$ (higher field), their varieties 
are in $\F_{2}^{|X_{PI}|}$ ($V(\langle rem_l\rangle)\subseteq \F_2^{|X_{PI}|}$) as they correspond to 
bit-level assignments to $X_{PI}$.
We have that $I(V(\langle rem_l\rangle)) = \langle rem_l \rangle + J_0$ (Theorem \ref{thm:strong-ns}).
Let $G_l=redGB(\langle rem_l \rangle + J_0 )$
denote the generators of the reduced \Grobner basis computation for a given $rem_l$.
% Then using Theorem~\ref{thm:coffF2}, $G_l = A \cdot \Fn[\xpi]$ where $A = \bigcap_{a\in V(G_l)}m_a$
% and $m_a$ is as defined in the theorem statement. 
% Since $G_l$ is computed using the Theorem~\ref{thm:elm},
% it is a Gr\"obner basis itself, and therefore, $E_L = A = \bigcap_{a\in \vpi(E_L)}m_a \subseteq \F_2[\xpi]$.
% In other words, the generators of the ideal $E_L$ computed using the Theorem~\ref{thm:chk} have coefficients in $\F_2$.
% The same result can also be shown for the generators of the ideals $E_H$ and $E'_H$. 
% Consequently, any ideal-interpolant $J_I$ computed
% for the pair $(E_L,E_H)$ will also have generators in $\F_2[\xpi]$. 
This, in turn implies that for 
any point in $\F_2^{|\xpi|}$, each generator of $G_l$ will either evaluate to 0 or 1.

% \begin{Example}
%   Let $p = 4/3 \cdot a_0 a_1 b_0 b_1 - 2 \cdot a_0 b_0 b_1 - 2/7 \cdot a_1 b_0$ be a polynomial in $R = \Q[a_0, a_1, b_0, b_1]$. Let $J_0 = \langle a_0^2 - a_0, \dots, b_1^2 - b_1 \rangle$. 
  
%   $redGB(\langle p + J_0 \rangle) = \{a_0 b_0 b_1, a_1 b_0\}$ 
%   % $a_0 a_1 b_0 b_1 + a_0 b_0 b_1 + a_1 b_0$
% \end{Example}

% We use this property to compute Boolean rectification functions. 
% Explain why this works: J + J0 / J0 - Von Neuman. Formula derived such that when each f_i vanishes, p vanishes. Whenever any f_i doesn't vanish, p = 1. When the rect fn evaluates to 0, we compute poly t which corresponds to rect fn. When rect fn should = 0, t = 0, when rect fn should = 1, t = 1. When rect fn = DC, t = 1 or 0. 



% % Rewrite, use examples. Or remove?
% The above result is used in our approach as follows: we use the polynomial $p$ to compute rectification functions. The points in $V(\langle f_1, \dots, f_s, J_0 \rangle) = $

% The product of maximal ideals 
% % corresponds to a pointwise interpolation, and hence
% is a canonical representation of the ideal $J+J_0$. 
% In addition, it is a property of such ideals $J+J_0$, that their reduced GB computed over $R_{J_0}$ result in generators with coefficients only in $\{-1,+1\}$. 
% Using these results, a rectification polynomial can be computed
% from the generators of $G_l$ and subsequently, a sub-circuit can be synthesized. 


% {\color{red}
% The ideal-interpolant $J_I$ is computed through a \Grobner basis which
% is a set of polynomials $G_I = \{g_1,\dots, g_t\}$. It is non-trivial
% to synthesize a single-output Boolean function from this set of
% polynomials. This problem can be simplified if we can compute a
% rectification  $\uc$ from $G_I$ with $V(\uc) = V(G_{I})$,
% and then synthesize it into a subcircuit.  
% %More specifically, we use the generators of $E_L$ for the computation. 
% %As our goal is to synthesize the rectification function $\uc$, 
% This can be achieved if the coefficients of $\uc$ are in
% $\F_2=\{0,1\}$, as $X_{PI}$ variables already take values in
% $\{0,1\}$. {\bf For this we first (wish to) prove the following
%   conjecture on the nature of generators of the ideal $E_L$.}

% \begin{Conjecture}
% \label{thm:ELF2}
% Ideal $E_L$ as given in Thm \ref{thm:chk} is computed as follows:
% \begin{enumerate}
% \item Compute $G=GB(J_L+J_0)$ with the lex order ``non-$\xpi$ variables $> \xpi$ variables''
% \item From $G$, obtain $G_L = G \cap\Fkk[X_{PI}] = \{g_1,\dots,g_t\}$
%   as the \Grobner basis of the elimination ideal $E_L$, i.e. $G_L = GB(E_L)$.
% \end{enumerate}
% Then the polynomials $g_1,\dots,g_t \in \F_2[\xpi]$.
% % Then from this $GB$ set, we pick the polynomials that contain only $\xpi$ variables.
% % We know that $\vpi(E_L) \subseteq \F_2^{|\xpi|}$, can we show that the 
% % coefficients of the generators of the ideal $E_L$, computed as above,
% % are from the set $\{0,1\}$. 
% \end{Conjecture} 

% {\bf
% Florian:  If we can show that the above conjecture is true, $\uc$ computed
%   from Eqn. \ref{eqn:id2poly} will evaluate to $\{0,1\}$, i.e in $\F_2$.
% As a result, $\uc$ can be synthesized using AND and XOR gates. In all
% our experiments, we have observed that the computed \Grobner basis is
% in $\F_2[\xpi]$. In subsection \ref{sec:conj_proof}, Utkarsh gives a
% reasoning for why we think this always happens, and his reasoning relies on
% the subsequent concepts. That is why it is described in two sections.
% A similar proof can also be shown for the generators of $E_H$ and
% $E'_H$. 
% }}

% \subsection{Obtaining $\uc$ from $G_l$}

% %% In finite fields, given an ideal $J$, it always possible to find a 
% %% polynomial $U$ such that $V(U) = V(J)$. The reason is that
% %% every ideal in a finite field has a finite
% %% variety and a polynomial with those points as its roots can always be
% %% constructed using the Lagrangian interpolation formula. We construct
% %% the rectification polynomial $U$ from the ideal-interpolant $E_L$ as
% %% shown below, such that $V(E_L) = V(U)$. 

% With $GB(E_L) = \{g_1,\dots,g_t\}$, $U$ is computed as:
% \begin{equation}
% \label{eqn:id2poly}
% U = (1+g_1)(1+g_2)\cdots(1+g_t)+1
% \end{equation}

% It is easy to assert that $V(U) = V(E_L)$. 
% Consider a point $\bm{a} \in V(E_L)$. As all of 
% $g_1,\dots,g_t$ vanish (= 0) at $\bm{a}$,
% \begin{align*}
% U(\bm{a}) &= (1+g_1(\bm{a}))(1+g_2(\bm{a}))\cdots(1+g_t(\bm{a}))+1 \\
% &= (1+0)(1+0)\cdots(1+0) + 1 = 0
% \end{align*}
% Conversely, for a point $\bm{a'} \not \in V(E_L)$, at least one  
% of $g_1,\dots,g_t$ will evaluate to 1. 
% Without loss of generality, if
% $g_1$ evaluates to 1 at $\bm{a'}$, then $U=(1+1)(1+0)\cdots(1+0)+1 = 1 \neq
% 0$. In addition to $\vpi(\uc) = \vpi(E_L)$, $\uc$ either evaluates to 0 or 1. 
% %%%%%%%%% The algorithm is commented out
% %% \par Using Eqn. (\ref{eqn:id2poly}), a recursive procedure is derived to
% %% compute $U$, and it is depicted in Algorithm \ref{algo:id2poly}. At
% %% every recursive step, we also reduce the intermediate  
% %% results by $\pmod{ J_0}$ (line 7) so as to avoid terms of high degree.
% In this fashion, from the ideal-interpolant $E_L$, we compute the
% single-fix rectification polynomial function $\uc$, and synthesize a
% sub-circuit at net $x_i$ such that $x_i = \uc$ rectifies the circuit. 
% % However, in~\cite{Utkarsh:VLSI18}, it was 
% % shown that it is a property of such ideals ($\langle rem_l, J_0 \rangle$) that their reduced
% % \Grobner bases (Def.~\ref{def:rgb}) have coefficients only in $\Ftwo$.
% Further, it was shown that, given an ideal $I$ with coefficients in $\Ftwo$ with generators  $\{g_1,\dots,g_t\}$, 
% a polynomial $p$ can always be constructed as $p = (1+g_1)(1+g_2)\dots(1+g_t)+1$, such that $V(p) = V(I)$. 
% This helps in computing a singleton polynomial for each 
%% {\it Ideal-Poly Conversion:} Over finite fields, given an ideal $J$, 
%% a polynomial $p$ can always be computed 
%% such that $V(p) = V(J)$. Let $\{g_1,\dots,g_t\}$ denote the generators
%% of $J$. Then, the polynomial $p$ can be constructed as, 
%% $p = (1+g_1)(1+g_2)\dots(1+g_t)+1$~\cite{Utkarsh:VLSI18}. 
% i.e. $V(rem_l)\subseteq \F_{2}^{|X_{PI}|}, 1 \leq l \leq 2^m$.
% It is a property of such ideals ($\langle rem_l,\jzxpi \rangle$), that their Reduced
% \Grobner basis (Def.~\ref{def:rgb}) will have coefficients only in $\Ftwo$.
% Once we compute a GB with coefficients in $\Ftwo$, we can translate it to a polynomial 
% which has the same variety as the ideal (Sec.~\ref{sec:prelim}).
We utilize the above facts and reduced GB computation to 
construct a synthesizable Boolean function $p_{G_l}$ from a polynomial function $rem_l$.
Consequently, the rectification function operations are restricted to
algebraic computations in $\F_2[X_{PI}]\equiv \B$. 

To compute the patch $u_i$, we perform the following steps:
\bi
\item Compute reduced \Grobner bases $G_l$ of $\langle rem_l, J_0 \rangle$ for each $rem_l$.
\item Construct a singleton polynomial $p_{G_l}$ such that \\ $V(p_{G_l}) = V(\langle rem_l, J_0 \rangle)$.
\item Impose an order on the sets for $DC_{pair}$ and composite set computations.
\item Compute $DC_{pair}$ using Eqn.~(\ref{eqn:dc_pair}), and then obtain the composite sets
      in Eqn.~(\ref{eqn:composite_dc}) which are then assigned to DC-, on- and off-sets of the 
      rectification functions (Sec.~\ref{comp:DFC1}).
\item In order to perform the variety union, intersection, and difference operations, we use ideal 
        product, sum, and colon operations, respectively, on the
        singleton polynomial $p_{G_l}$ representation of the ideal.
% Use ideal operations sum, product, and colon ideal to perform intersection, 
% sum, and difference of varieties, respectively.
\item The above procedure delivers $u_{i_{DC}}$ and $u_{i_{on}}$ as singleton polynomials in $\Ftwo[X_{PI}]$.
Translate $u_{i_{DC}}$ and $u_{i_{on}}$ into Boolean functions by interpreting the product, sum,
and '+1' as Boolean AND, XOR, and INV gates, respectively. Optimize the on-set $u_{i_{on}}$ w.r.t. to the DC-set $u_{i_{DC}}$.
\ei

%  A polynomial in $\F_2[X_{PI}]$ can be converted to a Boolean AND-XOR operation. This expression can be
% synthesized into a sub-circuit whose output can be connected to the respective targets in $W$ to rectify the circuit.
% As the product and sum operations are performed modulo 2; they can be implemented
% in a circuit using AND and XOR gates, respectively.

% The expression can be synthesized into a sub-circuit by replacing the modulo 2 product and sum 
% in the polynomial expression with the Boolean AND and XOR operators, respectively.
% We synthesize a subcircuit by interpreting sum as XOR $\pmod{2}$ and product 
% as AND gates.
% Subsequently,  we convert the polynomial to corresponding Boolean functions by treating $+,\cdot,+1$ 
% as XOR, AND, and INV operations, respectively.

% In order to compute a desired $u_i \in \F_2[X_{PI}]$ with a 
% $\F_2^{|X_{PI}|} \mapsto \F_2$ mapping,
% we perform the following steps.


% \bi
% \item Compute the generators $Gr_l$ for the ideal $Jr_l=\langle rem_l \rangle$ as 
% $redGB(Jr_l+\jzxpi)$, where $rem_l \in \Fkn[X_{PI}]$.
% Then the polynomials in $Gr_l$ have coefficients in $\F_2$,
% and variables in $\xpi$[Corollary~8.1]~\cite{UTK:thesis}.
% \item Use the Ideal-Poly conversion procedure from (Section \ref{sec:prelim})
% to compute a polynomial $rem_l'$. The computed $rem_l'$ is such that 
% $rem_l'\in \F_2[X_{PI}]$ and $V(rem_l') = V(rem_l)$.
% \item To compute $u_i$ algebraically, we utilize the relevant ideal operations described in
% the ideal-variety correspondences (Section \ref{sec:prelim}) to perform the union, intersection,
% and set difference of varieties over $V(rem_l')$.
% \ei


% $U_{dc} = \bigcap\limits_{l=1}^{2^m}V(pr_l)$

% \begin{Example}
% Continuing with our example~\ref{ex:4}, for the circuit shown in Fig.~\ref{fig:mas_bug_W},
% the intersection of Variety of remainders yields us the DC for all the targets.

% $U_{dc} = \bigcap\limits_{l=1}^{4}V(pr_l) = a_2b_1b_2+a_2b_1+a_2b_2+1$
% \end{Example}

% However, our experiments show that for a large class of benchmarks depending on the
% target selection, the intersection of all sets is most likely to be empty. Thus,
% we need a stronger DC set computation for individual targets.

% Recall that the variety of each $pr_l$ corresponds to the set of correct points
% for a given tuple assignment to the targets from $W_c[l]$. Thus, all 
% the points which lie in the intersection of all the $pr_l$'s comprises the 
% DC for all the targets.
% We can compute a polynomial $pr_l \in \F_{2}^{|X_{PI}|}$
%  such that $V(pr_l) = V(rem_l)$~\cite{Utkarsh:VLSI18}. 
% \subsubsection{DC for individual targets}

\begin{Example}\label{ex:4}
  Continuing with Ex.~\ref{ex:3}:
  % , we illustrate the above procedure:

  \bi
  \item $rem_3 = (\ga+1)a_2b_1b_2+a_2b_1 + (\ga^2+\ga)a_2b_2$ %\\ coefficients in $\Fkn$
  \item $G_3 = redGB(\langle rem_3,J_0\rangle) = \{a_2b_1,a_2b_2\}$ %\\ coefficients in $\Ftwo$
  \item $p_{G_3} = (1+a_2b_1)*(1+a_2b_2)+1 = a_2b_1b_2+a_2b_1+a_2b_2$
  \bi
    \item Here, $V(p_{G_3}) = V(\langle rem_3, J_0 \rangle)$
  \ei
  \item Similarly:
  \bi
    \item $p_{G_1} = a_2b_2$
    \item $p_{G_2} = a_2b_1b_2$
    \item $p_{G_4} = a_2b_1b_2 + a_2b_1$
  \ei
  \ei
  The rectification polynomials for the targets $(r_3,rr_3)$ computed using the greedy approach~(Sec.~\ref{comp:GFC})
  \begin{align*}
    % \begin{split}
      &u_{1_{on}}  = a_2b_1b_2; &u_{1_{off}} &= a_2b_1b_2 +1; &r_3  &= u_{1_{patch}} =(a_2\wedge b_1 \wedge b_2);\\
      &u_{2_{on}}  = a_2b_2;    &u_{2_{off}} &= a_2b_2+1;     &rr_3 &= u_{2_{patch}}= (a_2\wedge b_2).
    % \end{split}
  \end{align*}
  The rectification polynomials for the targets $(r_3,rr_3)$ computed using the on-set and don't care simplification~(Sec.~\ref{comp:DFC2})
  \begin{flalign*}
    % \begin{split}
     &u_{1_{dc}} = a_2b_1b_2+a_2b_2;&u_{1_{on}} &= a_2b_1b_2; &r_3 &= u_{1_{patch}} = a_2\wedge b_2; \\
     &u_{2_{dc}} = a_2b_2+1;        &u_{2_{on}} &= a_2b_2;    &rr_3 &= u_{2_{patch}} = 1. 
    % \end{split}
  \end{flalign*}
  % As seen above, the synthesis tool ({\it sis}) was able to simplify 
  % the on-set for $u_1$ patch function.
\end{Example}
