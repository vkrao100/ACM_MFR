\section{Introduction}
\par Debugging and rectification of digital logic circuits aims to correct
a given defective circuit implementation to match its intended
specification. As opposed to a complete redesign of the circuit, it
is desirable to synthesize rectification sub-functions with
minimal topological changes to the existing design -- a problem
often termed as {\it partial synthesis}. 
The process constitutes identifying
candidate nets in the circuit as potential targets for rectification, followed by  
a check to ascertain the rectifiability of the circuit at these targets. 
If the check confirms that the targets admit correction, corresponding rectification
functions are computed and synthesized to patch the circuit at these targets.
%  If the targets admit correction, corresponding rectification functions
% are computed in terms of primary inputs, followed by re-basing wherein the 
% computed function is synthesized in terms of prior design efforts}.
%Begin ACM journal
  
It is akin to performing synthesis for Engineering Change Order 
(ECO), wherein a highly optimized implementation is minimally modified to match the updated specification in a cost-effective way. This is achieved by reusing prior design efforts and avoiding rerunning the entire synthesis flow while adhering to the resource constraints and the physical design limitations.

The rectification problem has witnessed much research over the years -- 
some of the earliest being~\cite{Sadowska:DAC95,scholl:1,andreas:2005}.
Owing to a manifold improvement in the efficiency of SAT solvers,
there has been a renewed interest in the problem over the last decade from 
the logic synthesis, testing, and verification communities~\cite{
MF_Huang:DATE12,scholl:2,SS_Fujita:ISQED17,SS_Alan:DAC18}.
These techniques generally employ SAT, Quantified Boolean Formula (QBF) solving,
and Craig Interpolation (CI) based techniques for rectification. While
successful for control-dominated applications, these techniques are
computationally infeasible for the rectification of arithmetic circuits.
Symbolic Computer Algebra (SCA) techniques are found to be more
suitable for formal analysis and verification of arithmetic circuits.
However, utilization of the various facets and capabilities of the SCA
techniques for post-verification debugging and rectification has only
recently begun to be addressed
\cite{farimah:2017:1,MF_Rolf:ISVLSI18,Utkarsh:VLSI18,
Vkrao:FMCAD18,Utkarsh:ETS19,Vkrao:ISQED21,Vkrao:GLSVLSI21}. 
%End ACM journal



\subsection{Motivation}
This article addresses the problem of {\it multi-target rectification
of faulty finite field arithmetic circuits}.
Such circuits find applications in cryptography,
error-control codes, RFID tags, testing of VLSI circuits, among others. Specifically, Elliptic Curve Cryptography (ECC) is one of the most important usage models where the public-key cryptography is designed on the algebraic structure of elliptic curves over finite fields. Due to its shorter key length and efficiency ~\cite{ecc_app:2016}, 
ECC is fast becoming the encryption standard with applications in cryptocurrency transaction signing and securing web traffic.
The main building blocks of ECC hardware implementations 
are fast, custom-built finite field arithmetic circuits, 
thus raising the potential for errors in the implementation, 
which have to be eventually rectified. 
It was shown~\cite{crypto:bug_attacks} that incorrect cryptography 
hardware can lead to full leakage of the security key and even
counterfeiting~\cite{crypto:counterfeit}.
To secure data privacy and avoid vulnerabilities in crypto-systems, their
rectification is imperative.


%End ACM journal
%While successful for control-dominated applications, the SAT-based 
%models~[4-10], however, are infeasible for rectification of arithmetic circuits. 

% {\it Problem Statement and Objective:}
\subsection{Problem Statement and Objective}
We are given the following: 
\bi
\item As the specification, a multivariate
polynomial $f$ with coefficients in a finite field of $2^n$ elements
(denoted $\F_{2^n}$), for a given  $n\in \Z_{> 0}$.
\item An irreducible
polynomial $P_n(x)$ of degree $n$ with coefficients in $\{0,1\}$ used 
to construct $\Fkn$.
\item A faulty circuit implementation $C$,
with no assumptions on the number or the type of bugs present in
$C$. 
\item A set $W = \{w_1,\dots,w_m\}$ of $m$ candidate targets from $C$,
pre-specified or selected using contemporary signal selection heuristics 
~\cite{SS_Alan:DAC18,SS_Fujita:ISCAS19,SS_Roland:DAC19}.
\ei
% We further assume that it has been ascertained that $C$ 
% admits rectification at these $m$ targets, using~\cite{MF_Huang:DATE12,Vkrao:ISQED21}. 
The objective of our approach is to: 
\bi
 \item Ascertain that $C$ 
admits rectification at these $m$ targets.
\item Compute a set
of individual rectification functions $U =
\{u_1,\dots,u_m\}$ for the corresponding targets. 
\item Derive {\it don't care conditions}
corresponding to the $m$ rectification functions. 
\item Synthesize
the rectification polynomials into logic sub-circuit patches.
%Note thatn the cardinality of $|W|=|U|=|O|=m$.
\ei

\subsection{Approach and Contribution}
% In particular, we exploit the ideal-variety
% correspondences to explore and compute rectification functions for the $m$-targets
% collectively. 
We model the rectification problem using concepts from algebraic geometry and
use symbolic computer algebra algorithms to determine rectifiability and to 
compute rectification functions. 
The specification $f$ and implementation $C$ are modeled in terms of polynomial ideals. 
The rectifiability check is formulated on these ideals using the Strong Nullstellensatz over finite fields. Subsequently, 
the rectification functions are computed using the \Grobner bases~\cite{gb_book} of these ideals.
In this regard, our approach goes beyond the Nullstellensatz-based 
results produced from the $m$-target rectifiability check presented 
in~\cite{Vkrao:ISQED21}, and {\it computes} $m$ individual rectification 
functions, altogether. We show how, in our algebraic model the don't cares correspond to {\it varieties of polynomial ideals}, and how they can be computed with \Grobner bases. The rectification functions are computed as polynomials whose common zeros correspond to the minterms of the functions.
Overall, our contributions are as follows:

\bi
\item Formulate a check to determine the rectifiability of $C$ at
the $m$ targets.
\bi
    \item This check implies the existence of a polynomial function over $\Fkn$ with
    mapping $u_i:\F_2^{|X_{PI}|}\rightarrow\F_2$ at individual targets $w_i$. Here, 
    $1 \leq i \leq m$, $\F_2=\{0,1\}$, and $X_{PI}$ denotes the set of primary input variables of $C$. 
    \item Substituting the polynomial function at the corresponding targets, $w_i = u_i[X_{PI}]$, rectifies the circuit.
\ei

\item While there may exist multiple rectification functions for each of the $m$ targets, we compute one such individual function for each of the targets. 
We present two approaches for patch function computation:

\bi 
\item First, a heuristic which greedily resolves the rectification functions 
for the targets.
\item A second heuristic which explores a subset of {\it don't care conditions} for the targets.
\bi
    \item We present efficient techniques to explore the space of various admissible rectification
functions, in turn, computing subsets of don't care conditions
which helps in simplifying the rectification patches.

    \item Synthesis of the corresponding polynomial patch functions along with don't cares, demonstrates the efficacy of our approach in terms of improved area and delay characteristics of the patches.

\ei
\ei

\item  We further propose a synthesis
procedure that can translate such polynomial functions to Boolean rectification 
functions. Subsequently, these functions are translated to Boolean logic by converting the algebraic multiplication and addition operations to the Boolean AND and XOR operations, respectively.

\item We present theoretical concepts and algorithms, and their implementations, 
to provide a scalable and efficient solution for the rectification of finite field arithmetic circuits.
\bi
\item In this regard, we show how to efficiently represent and manipulate our multi-target rectification formulation on circuits using the Boolean data-structure of Zero-suppressed Binary Decision Diagrams (ZDDs).
\item We utilize ZDDs to perform the core GB-based reduction computation.
\ei

\item We substantiate the efficacy of our techniques by rectifying 
large operand-width finite field arithmetic circuits, where conventional SAT-solver based 
rectification approaches are infeasible.
\bi

\item We reason about the presence or absence of solutions and other properties of a system of polynomials without explicitly solving them. 
In contrast, the contemporary approaches explicitly solve for a solution at each step and hence are infeasible towards rectification of arithmetic circuits.
\ei

 
\ei



% Our approach is implemented using the polynomial algebra computational
% engines of {\sc Singular} \cite{DGPS_410} and {\sc PolyBori}\cite{pbori:JSC09}.

% Derive the necessary and sufficient conditions
% for the existence of a rectification function at multiple targets.
% We compute logic corrections as polynomial functions in primary input variables
% for the given targets, which are synthesized into rectification patches for the corresponding targets.
% We introduce the notion of word-level don't 
% cares in a polynomial algebra setting.


%% However, exploiting symbolic algebra techniques on generating low-cost patches and
%% computing a patch function in terms of internal nets needs further investigation and 
%% tuning, which is beyond the scope of this paper.

The article is organized as follows: The following section covers preliminary
background. Section~\ref{sec:pmodel} reviews the polynomial modeling
concepts. Rectification check formulation is described in
Section~\ref{sec:rcheck}, followed by the rectification function and
don't care computations in Section~\ref{sec:rfunc}. Section~\ref{impl}
discusses the implementation details, and experimental
results are described in Section \ref{sec:exp}, and
Section~\ref{sec:conc} concludes the article. 

\section{Review of Prior Work}

Contemporary approaches formulate rectification
using QBF solving~\cite{scholl:2}, using CI or iterative SAT
solving~\cite{MF_Roland:ICCAD10,MF_Huang:DATE12}.  
The rectification techniques in 
\cite{fujita:2015,SS_Fujita:ISCAS19,MF_Huang:DATE12,SS_Roland:DAC18} 
iteratively and incrementally compute multiple single-fix functions
that partially patch the circuit in each iteration.
 They ensure that, in each iteration, erroneous minterms are resolved
and no new errors are introduced, eventually converging the circuit
to the given specification.
Recent techniques incorporate resource awareness in 
patch generation by re-expressing the obtained Skolem functions 
in terms of internal signals~\cite{SS_Alan:DAC18}, employ improved heuristics for
target selection~\cite{SS_Fujita:ISCAS19}, or resolve a combination of
such objectives, such as the symbolic sampling approach of~\cite{SS_Roland:DAC19}.
 In~\cite{SS_Roland:DAC19} the authors propose a robust ECO approach to derive 
patches with minimal impact on the heavily optimized existing implementation 
against a structurally dissimilar ECO-evolved specification. 
They enumerate rectification points functionally by simulation and match the circuitry of patches implicitly to maximize the reuse of existing logic in the implementation. To achieve 
scalability, the method proposes modeling and analyzing its computations 
in symbolic sampling domain. However, circuits that implement polynomial 
computations over large bit-vector operands are hard to rectify using 
models based on Boolean function, SAT/SMT-solvers, etc.
Our experiments show that the contemporary SAT solvers fail to rectify 
finite field circuits beyond 16-bit operands.

% While successful for control-dominated applications, these techniques are
% computationally infeasible for rectification of arithmetic circuits.

\subsection{Symbolic Computer Algebra}

In the context of arithmetic circuits, symbolic computer algebra 
techniques for integer arithmetic~\cite{farimah:2016:1,farimah:2017:1,
MF_Rolf:ISVLSI18} and finite field circuits~\cite{Utkarsh:ETS19,
Utkarsh:VLSI18,Vkrao:FMCAD18} have been considered for rectification. 
The rectification approaches presented in~\cite{farimah:2016:1,farimah:2017:1} 
rely heavily on the structure of the arithmetic circuit. Further,
if the circuit contains redundancies, their approach 
fails~\cite{farimah_cex} to resolve the rectification question. 
Further, the approach of~\cite{MF_Rolf:ISVLSI18} is limited by the fault model,
which addresses only a gate misplacement fault.
The authors in~\cite{Utkarsh:ETS19,Utkarsh:VLSI18} present a Weak 
Nullstellensatz based rectification formulation to determine the existence of 
rectification functions and solve them using GB-based techniques. 
Subsequently, the rectification function is computed by the application of 
Craig Interpolation in polynomial algebra over finite fields. 
The authors show how the lattice of all algebraic interpolants 
correspond to don't care conditions for the rectification function, and synthesize
rectification functions from algebraic interpolants.
As an alternative to CI, in~\cite{Vkrao:FMCAD18} we presented an approach where
rectification was formulated using ideal membership and extended GB theory.
% The authors introduce a complete theory of Craig Interpolation in polynomial
% algebra over finite fields using Hilbert's Nullstellensatz, and device GB-algorithms
% to compute them.
However, all these algebraic approaches address only {\it single-fix 
rectification} -- where irrespective of the type or number of bugs in the circuit, rectification is attempted at a single net. This is too restrictive, and depending on the nature of the bugs, the circuit may not admit single-fix rectification at all. In such cases, 
the correction has to be attempted at multiple targets.

Recently, we proposed a word-level SCA based approach~\cite{Vkrao:ISQED21} 
to {\it decide multi-target rectifiability} in finite field circuits. 
Given a set of $m$-targets within a $n$-bit operand width circuit, 
the approach presents a Strong Nullstellensatz based decision procedure
to determine the existence of rectification functions at these targets.
The efficiency of the approach is derived by interpreting the $m$-targets
as a $m$-bit-vector. The problem was modeled at a word-level to enable the use of 
higher level abstraction in rectification and synthesis. 
However, enabling such word-level reasoning might induce 
computation issues across elements from the $n$-bit-vector space (circuit) 
and the $m$-bit-vector space (rectification patch). To resolve this
incompatibility, we presented theory and mathematical derivations that
facilitate algebraic computations over a unified domain modeled over 
$k=LCM(n,m)$. However, for finite field circuits, with practical applications 
in cryptography, the operand width $n$ as dictated by the National 
Institute of Standards and Technology (NIST) is generally a prime number. 
For example, NIST-ECC recommends $n$ to be 163 or larger 
($n$= 163, 233, 283, 409, 571, etc.). Thus, for any given target size $m$, 
$k=LCM(n,m)=n*m$ becomes extremely large. This results in algebraic computations 
involving substantially large primitive elements and constants.
The large size coupled with the complicated arithmetic nature of these 
circuits increases the complexity of the rectification problem.
Moreover, the proposed approach~\cite{Vkrao:ISQED21} can only ascertain whether {\it there exists} a set of 
functions that can patch the circuit at the targets. As it is only a decision procedure, the 
approach cannot {\it compute} rectification functions. Thus the problem 
of multi-fix rectification of data-path circuits remains unsolved, and 
theoretical and algorithmic solutions to compute and synthesize 
rectification patches for arithmetic circuits are still desired.