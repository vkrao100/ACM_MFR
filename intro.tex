\section{Introduction}
\par Debugging and rectification of digital logic circuits aims to correct
a given defective circuit implementation to match its intended
specification. {\red As opposed to a complete redesign of the circuit, it
is desirable to synthesize rectification sub-functions with
minimal topological changes to the existing design -- a problem
often termed as {\it partial synthesis}.} 
The process constitutes identifying
candidate nets in the circuit as targets for rectification, followed by  
a check to determine whether the circuit can be patched at these
targets. 
If the targets admit correction, corresponding rectification
functions are computed and synthesized to fix the circuit at these targets.
% {\red If the targets admit correction, corresponding rectification functions
% are computed in terms of primary inputs, followed by re-basing wherein the 
% computed function is synthesized in terms of prior design efforts}.
%Begin ACM journal
{\red  
It is akin to performing synthesis for Engineering Change Order 
(ECO), wherein a highly optimized implementation is minimally modified to match the 
updated specification in a cost effective way. This is achieved by reusing prior design 
efforts and avoiding rerunning the entire synthesis flow, while adhering to the resource 
constraints and the physical design limitations.

The rectification problem has witnessed a lot of research over the years -- 
some of the earliest being~\cite{Sadowska:DAC95,scholl:1,andreas:2005}.
Owing to a manifold improvement in the efficiency of SAT solvers,
there has been a renewed interest in the problem over the last decade from 
the logic synthesis, testing and verification communities~\cite{
MF_Huang:DATE12,scholl:2,SS_Fujita:ISQED17,SS_Alan:DAC18}.
These techniques generally employ SAT, Quantified Boolean Formula (QBF) solving,
and Craig Interpolation (CI) based techniques for rectification. While
successful for control-dominated applications, these techniques are
computationally infeasible for rectification of arithmetic circuits.
Symbolic Computer Algebra (SCA) techniques are found to be more
suitable for formal analysis and verification of arithmetic circuits.
However, utilization of the various facets and capabilities of the SCA
techniques for post-verification debugging and rectification has only
recently begun to be addressed
\cite{farimah:2017:1,MF_Rolf:ISVLSI18,Utkarsh:VLSI18,
Vkrao:FMCAD18,Utkarsh:ETS19,Vkrao:ISQED21,Vkrao:GLSVLSI21}. }
%End ACM journal


{\red
\subsection{Motivation}
This paper addresses the problem of {\it rectification
of faulty finite field arithmetic circuits} at a given set of
targets. Such circuits find applications in cryptography,
error-control codes, RFID tags, testing of VLSI circuits, among 
others. Specifically, Elliptic Curve Cryptography (ECC) is one of 
the most important usage models where the public-key cryptography 
is designed on the algebraic structure of elliptic curves over 
finite fields. Due to its shorter key length and efficiency ~\cite{ecc_app:2016}, 
ECC is fast becoming the encryption standard with applications 
in cryptocurrency transaction signing and securing web traffic.
Arithmetic circuits are mostly custom designed, thus 
raising the potential for errors in the implementation, 
which have to be eventually rectified. 
It was shown~\cite{crypto:bug_attacks} that incorrect cryptography 
hardware can lead to full leakage of the security key and even
counterfeiting~\cite{crypto:counterfeit}.
To secure data privacy and avoid vulnerabilities in crypto-systems, their
rectification is of utmost importance.}


%End ACM journal
%While successful for control-dominated applications, the SAT-based 
%models~[4-10] however are infeasible for rectification of arithmetic circuits. 

% {\it Problem Statement and Objective:}
\subsection{Problem Statement and Objective}
We are given the following: 
\bi
\item As the specification, a multivariate
polynomial $f$ with coefficients in a finite field of $2^n$ elements
(denoted $\F_{2^n}$), for a given  $n\in \Z_{> 0}$.
\item An irreducible
polynomial $P_n(x)$ of degree $n$ with coefficients in $\{0,1\}$ used 
to construct $\Fkn$.
\item A faulty circuit implementation $C$,
with no assumptions on the number or the type of bugs present in
$C$. 
\item A set $W = (w_1,\dots,w_m)$ of $m$ targets from $C$,
pre-specified or selected using contemporary signal selection heuristics 
~\cite{SS_Alan:DAC18,SS_Fujita:ISCAS19,SS_Roland:DAC19}.
\ei
% We further assume that it has been ascertained that $C$ 
% admits rectification at these $m$ targets, using~\cite{MF_Huang:DATE12,Vkrao:ISQED21}. 
The objective of our approach is to: 
\bi
{\red \item Ascertain that $C$ 
admits rectification at these $m$ targets}.
\item Compute a set
of individual rectification functions $U =
(u_1,\dots,u_m)$ for the corresponding targets. 
\bi 
\item Here, each $u_i$ is a 
polynomial function $u_i:\F_2^{|X_{PI}|}\rightarrow\F_2$, where $\F_2=\{0,1\}$, 
and $X_{PI}$ denotes the set of primary inputs.
\ei
\item Derive {\it don't care conditions}
corresponding to the $m$ rectification functions. 
\item Synthesize
the rectification polynomials into logic sub-circuit patches.
%Note that the cardinality of $|W|=|U|=|O|=m$.
\ei

\subsection{Prior Work}

Contemporary approaches formulate rectification
using QBF solving~\cite{scholl:2}, using CI or iterative SAT
solving~\cite{MF_Roland:ICCAD10,MF_Huang:DATE12}.  
The rectification techniques in 
\cite{fujita:2015,SS_Fujita:ISCAS19,MF_Huang:DATE12,SS_Roland:DAC18} 
iteratively and incrementally compute multiple single-fix functions
that partially patch the circuit in each iteration.
{\red They ensure that, in each iteration, erroneous minterms are resolved
and no new errors are introduced, eventually converging the circuit
to the given specification.}
The more recent techniques further
include more resource awareness in patch generation by re-expressing the obtained Skolem functions 
in terms of internal signals~\cite{SS_Alan:DAC18}, employ improved heuristics for
target selection~\cite{SS_Fujita:ISCAS19}, or resolve a combination of
such objectives, such as the symbolic sampling approach of~\cite{SS_Roland:DAC19}.
{\red In~\cite{SS_Roland:DAC19} the authors propose a robust ECO approach to derive 
patches with minimal impact on the heavily optimized existing implementation 
against a structurally dissimilar ECO-evolved specification. 
They enumerate rectification points 
functionally by simulation and match the circuitry of patches implicitly 
to maximize reuse of existing logic in the implementation. To achieve 
scalability, the method proposes modeling and analyzing its computations 
in symbolic sampling domain. However, circuits that implement polynomial 
computations over large bit-vector operands are hard to rectify using 
models based on Boolean function, SAT/SMT-solvers, etc.
Our experiments show that the contemporary SAT solvers fail to rectify 
finite field circuits beyond 16-bit operands.}

% While successful for control-dominated applications, these techniques are
% computationally infeasible for rectification of arithmetic circuits.

\subsubsection{Symbolic Computer Algebra}
{\red
In the context of arithmetic circuits, symbolic computer algebra 
techniques for integer arithmetic~\cite{farimah:2016:1,farimah:2017:1,
MF_Rolf:ISVLSI18} and finite field circuits~\cite{Utkarsh:ETS19,
Utkarsh:VLSI18,Vkrao:FMCAD18} have been considered for rectification. 
The rectification approaches presented in~\cite{farimah:2016:1,farimah:2017:1} 
rely heavily on the structure of the arithmetic circuit. Further,
if the arithmetic circuits contain redundancies, then their approach 
fails~\cite{farimah_cex} to resolve the rectification question. 
The authors in~\cite{Utkarsh:ETS19,Utkarsh:VLSI18} present a Weak 
Nullstellensatz based rectification formulation to determine existence 
of rectification functions and solve them using GB-based techniques. 
Subsequently, rectification function is computed by application of 
Craig Interpolation in polynomial algebra over finite fields. 
However, all these algebraic approaches address only {\it single-fix 
rectification} -- where irrespective of the type or number of bugs 
in the circuit, rectification is attempted at a single net. This is 
too restrictive and depending on the nature of the bugs, the circuit 
may not admit single-fix rectification at all. In such cases, 
correction has to be attempted at multiple targets.

Recently, we proposed a word-level SCA based approach~\cite{Vkrao:ISQED21} 
to {\it decide multi-target rectifiability} in finite field circuits. 
Given a set of $m$-targets within a $n$-bit operand width circuit, 
the approach presents a Strong Nullstellensatz based decision procedure
to determine existence of rectification functions at these targets.
The efficiency of the approach is derived by interpreting the $m$-targets
as a $m$-bit-vector. Enabling such a word-level reasoning might induce 
computation issues across elements from the $n$-bit-vector space (circuit) 
and the $m$-bit-vector space (rectification patch). To resolve this
incompatibility, we presented theory and mathematical derivations that
facilitate algebraic computations over a unified domain modeled over 
$k=LCM(n,m)$. However, for finite field circuits, with practical applications 
in cryptography, the operand width $n$ as recommended by the National 
Institute of Standards and Technology (NIST), is generally a prime number. 
For example, NIST-ECC recommends $n$ to be 163 or larger 
($n$= 163,233,283,409,571, etc.). Thus, for any given target size $m$, 
$k=LCM(n,m)=n*m$ becomes extremely large. This results in algebraic 
computations involving extremely large primitive elements and constants.
The large size coupled with the complicated arithmetic nature of these 
circuits increases the complexity of the rectification problem.
Moreover, the proposed approach~\cite{Vkrao:ISQED21} can only ascertain 
whether {\it there exists} a set of patch functions that can patch the 
circuit at those targets. As it is only a decision procedure, the 
approach cannot {\it compute} rectification functions. Thus the problem 
of multi-fix rectification of data-path circuits remains unsolved, and 
theoretical and algorithmic solutions to compute and synthesize 
rectification patches for arithmetic circuits are still desired.}

\subsection{Approach and Contribution}
% In particular, we exploit the ideal-variety
% correspondences to explore and compute rectification functions for the $m$-targets
% collectively. 
{\red Our techniques and algorithms are based on symbolic computer algebra
and algebraic geometry -- particularly on the concepts of the Strong
Nullstellensatz and \Grobner bases \cite{gb_book}. Rectification is 
performed against a given polynomial specification over finite fields, 
with the specification and implementation modeled in terms of polynomial 
ideals~\cite{gb_book}. The rectifiability check is formulated on these 
ideals using the Strong Nullstellensatz over finite fields. Subsequently, 
the rectification functions are computed using the \Grobner bases of these ideals. 
We model the rectifiability check as a decision procedure and the rectification 
function computation as a quantification procedure. 
In this regard, our approach goes beyond the Nullstellensatz-based 
results produced from the $m$-target rectifiability check presented 
in~\cite{Vkrao:ISQED21}, and {\it computes} $m$ individual rectification 
functions altogether. While there may exist multiple rectification functions 
for each of the $m$ targets, we compute and synthesize one such individual 
function for each of the targets. Further, we show how in our algebraic model 
the don't cares correspond to {\it varieties of polynomial ideals}, and how 
they can be computed with \Grobner bases.
We reason about the presence or absence of solutions 
and other properties of a system of polynomials without explicitly solving them. 
In contrast, the contemporary approaches explicitly solve for a solution at 
each step and hence are infeasible towards rectification of arithmetic circuits.
% Our approach is implemented using the polynomial algebra computational
% engines of {\sc Singular} \cite{DGPS_410} and {\sc PolyBori}\cite{pbori:JSC09}.

\subsubsection{Contributions}:
\bi
\item We derive the necessary and sufficient conditions
for the existence of a rectification function at multiple targets.

\item We compute logic corrections as polynomial functions in primary input variables
for the given targets, which are synthesized into rectification patches for the corresponding targets.

\item We introduce the notion of word-level (or multi-target??) don't 
cares in a polynomial algebra setting.

\item We present efficient techniques to explore the space of various admissible rectification
functions, in turn, computing subsets of don't care conditions
which help in simplifying the rectification patches.
	\bi
	\item Synthesis of the corresponding polynomial patch functions, along with don't cares, demonstrates 
	the efficacy of our approach in terms of improved area and delay characteristics of the patches.
	\ei

\item We present theoretical concepts and algorithms, and their implementations, 
to provide a scalable and efficient solution for rectification of finite field arithmetic circuits.

\item We substantiate the efficacy of our techniques by rectifying 
large operand-width finite field arithmetic circuits, where conventional SAT-solver based 
rectification approaches are infeasible.

\ei

%% However, exploiting symbolic algebra techniques on generating low-cost patches and
%% computing a patch function in terms of internal nets needs further investigation and 
%% tuning, which is beyond the scope of this paper.

{\it Paper Organization:} The following section covers preliminary
background. Section~\ref{sec:pmodel} reviews the polynomial modeling
concepts. Rectification check formulation is described in
Section~\ref{sec:rcheck}, followed by the rectification function and
don't care computations in Section~\ref{sec:rfunc}. Section~\ref{impl}
discusses the implementation details, and experimental
results are described in Section \ref{sec:exp}, and
Section~\ref{sec:conc} concludes the paper. 