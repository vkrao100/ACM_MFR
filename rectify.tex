\section{Rectification Check}\label{sec:rcheck}

% The rectifiability check presented in~\cite{Vkrao:ISQED21} relies on the 
% results of the Strong Nullstellensatz over finite fields and \Grobner
% basis reduction $f\xrightarrow{GB(J+J_0)}_+ r$. 
% To help formulate the rectification function computation, we restate 
% the decision procedure presented in~\cite{Vkrao:ISQED21} as per our needs.
In~\cite{Vkrao:ISQED21}, the authors presented techniques 
which utilize the aforementioned polynomial ideal setup
to derive the necessary and sufficient conditions for the existence
of a multi-fix rectification at the given set of targets.
We rephrase and restate Thm. V.1 from~\cite{Vkrao:ISQED21}, and 
briefly discuss its key aspects. 
Subsequently, we formulate the computation of rectification functions by utilizing
the outcome of their decision procedure.
% Subsequently, we formulate the computation of rectification functions as a 
% quantification procedure by building upon
% the outcome of their decision procedure.

\begin{Theorem}{\bf [Multi-fix Rectification Theorem]}\label{Thm:rect}
A \spec~polynomial $f$, a faulty 
\impl~$C$ represented using the ideal 
$J +J_0 = \langle F \cup F_0\rangle \subset R$,
 and a set of targets $W=(w_1,\dots,w_m) \subset \{X-X_{PI}\}$
 are given. 
%  Here, each $w_i$, for $i=1,\dots,m$ represents the output
% of an $i^{th}$ gate in $C$.
% The targets in set $W$ are considered fan-in free or treated as pseudo primary inputs.
RTTO $>$ is imposed on $R$. Let $W_c = \{(0,0,..,0),\dots,(1,1,..,1)\},$ $~|W_c| = 2^m$, 
 denote the set of all possible $\{0,1\}$ assignments to targets $W$.
 This is akin to computing cofactors of the circuit functions with 
 respect to the targets $W$.
 % Roland et al.~\cite{MF_Huang:DATE12} refer to them as cofactors
 % and we use the same terminology. 
Each cofactor tuple $W_c[l]$ serves as one set
 of assignments to $m$ targets at their respective indexes in $W$. 
% We construct the ideals by considering all the
% t 
% Here $F=\{f_1,\dots,f_{w_1}:w_1+tail(f_{w_1}),\dots,f_{w_m}:w_m+tail(f_{w_m}),\dots,f_s\}$
%(Setup as described in Sec.\ref{sec:rsetup}),
The following ideals are constructed:  
\bi
\item {\small $J_l = \langle F_l\rangle =\langle f_1,\dots,f_{w_1}:w_1+W_c[l][1],
	\dots,f_{w_m}:w_m+W_c[l][m]$ $,\dots,f_s\rangle$}, $\forall l \in 1,\dots 2^m$. 
\ei

Reduce $f$ by $F_l\cup F_0$ to obtain remainders $rem_l$: 
$f\xrightarrow{F_l\cup F_{0}}_+ rem_l,$  for $1 \leq l \leq 2^m$.
Then, the circuit $C$ is rectifiable at the target set $W$ {\bf\textit{if
   and only if}} union of varieties $\bigcup\limits_{l=1}^{2^m}V(rem_l) = \Ftwo^{|X_{PI}|}$. 
\end{Theorem}

{\red
To synthesize a patch, a desired $u_i$ corresponds
to a polynomial function $u_i:\F_2^{|X_{PI}|} \mapsto \F_2$. 
\begin{Proof}
As the correction at target $W$ makes $C$ match $f$, $f$ should vanish on
$V_{\Fkk}(J')$.
Moreover, each $rem_l$ comprises only $X_{PI}$ variables. This is
because WRTO $>_R$ ensures that each non primary input variable (each gate
output and  word-level variable) appears as the leading term of some
polynomial in $F'$. Thus each non primary input variable is canceled
in the reduction $f\xrightarrow{F'_l, F'_{0}}_+ rem_l$. Furthermore,
as $X_{PI}$ take values in $\F_2$, $x^2=x, \forall x \in
X_{PI}$. Hence, 
% even though $V_{\Fkk}(J')$ is evaluated in $\Fkk$,
$V(rem_l) \subseteq \F_{2}^{|X_{PI}|}$. Thus, the rectification theorem
 can be equivalently stated as: ``$f$ vanishes on
\begin{small}
$V_{\Fkk}(J') \iff \bigcup\limits_{l=1}^{2^m}V(rem_l) = \Ftwo^{|X_{PI}|}$''.
\end{small} 

(i) {\bf To prove ``$\Rightarrow$''}: Let $x_{PI} \in \Ftwo^{|X_{PI}|}$ be an
assignment to the primary input variables of $C$. Every assignment
$x_{PI}$ results in a corresponding assignment $x_{int}$ 
to rest of the variables in $C$. For each such point $(x_{PI},x_{int})\in \Fkk$,
the target $W$ evaluates to one of the values in the list $\delta$,
i.e. $(0,1,\be,\dots,\beta^{2^m-2})$. When $W = 0$, $J'_1$ vanishes on
the point $(x_{PI},x_{int})$. Likewise, $J'_2$ vanishes on
$(x_{PI},x_{int})$ when $W = 1$, and so on. Since
$f\xrightarrow{F'_l\cup F'_0}_+rem_l,1 \leq l \leq 2^m$, and $f$ vanishes
on the point $(x_{PI},x_{int})$ to begin with, we obtain that for
every  primary input assignment $x_{PI}$, one of the $rem_l$ vanishes. This
implies that $ \bigcup\limits_{l=1}^{2^m}V(rem_l) = \Ftwo^{|X_{PI}|}$.

(ii) {\bf To prove ``$\Leftarrow$''}: Say there exists an assignment to the
primary inputs $x_{PI} \in \Ftwo^{|X_{PI}|}$ such that $rem_1$ vanishes on
$x_{PI}$, i.e. $rem_1(x_{PI})=0$. For the given point $x_{PI}$, the rest of the variables 
of $C$ get a corresponding assignment $x_{int}$. 
As $f\xrightarrow{F'_1\cup F'_0}_+ rem_1$, we have that $f$ is a member of the
ideal $J'_1 + J'_0 + \langle rem_1 \rangle$. Therefore, when
$rem_1(x_{PI})=0$, the ideal $J'_1$ also vanishes on $(x_{PI},x_{int}) \in \Fkk$
because the tuple $(x_{PI},x_{int})$ is a valid evaluation of the circuit.
Further, $J'_0$ by definition vanishes everywhere in $R'$. This implies that
$f(x_{PI},x_{int})=0$. The argument similarly holds for each
$rem_{l}$ vanishing on some $x_{PI}$. This proves that for all primary
inputs, if any $rem_l:1 \leq l \leq 2^m$ vanishes, then $f$ vanishes too; and 
that completes the proof.
\end{Proof}
 }

The above check for union of varieties can be performed 
% as $\prod_{l=1}^{2^m} rem_l\xrightarrow{J_{0}^{X_{PI}}}_+0?$ (cf. Section \ref{sec:prelim}).
as product of ideals, i.e. by checking if $\prod_{l=1}^{2^m} 
rem_l\xrightarrow{J_{0}}_+0$.
RTTO $>$ is known to have the property that makes the division 
$f\xrightarrow{F_l\cup F_{0}}_+ rem_l$ mimic gate level substitution
in polynomial algebra.
% RTTO $>$, $f\xrightarrow{F_l\cup F_{0}}_+ rem_l$ 
% mimics polynomial substitution. 
Thus, after reduction, 
all non-primary input variables in the circuit are canceled and the 
final remainder has only $X_{PI}$ variables in its support.

\begin{Example}
\label{ex:3}
{\it 
Continuing on with the Ex. \ref{verify_ex}, we
demonstrate the rectification check presented in~\cite{Vkrao:ISQED21}
for $W=(r_3,rr_3)$. 

Constructing the $J_l$ ideals:
\bi
\item {\small$J_1 = \langle F_1\rangle$, where $F_1[f_{26}: r_3+0],F_1[f_{27}: rr_3 + 0]$},$(r_3 =0, rr_3 = 0)$ 
\item {\small$J_2 = \langle F_2\rangle$, where $F_2[f_{26}: r_3+0],F_2[f_{27}: rr_3 + 1]$},$(r_3 =0, rr_3 = 1)$
\item {\small$J_3 = \langle F_3\rangle$, where $F_3[f_{26}: r_3+1],F_3[f_{27}: rr_3 + 0]$},$(r_3 =1, rr_3 = 0)$
\item {\small$J_4 = \langle F_4\rangle$, where $F_4[f_{26}: r_3+1],F_4[f_{27}: rr_3 + 1]$},$(r_3 =1, rr_3 = 1)$
\ei
Reducing the $\spec$ $f: Z+A\cdot B$ modulo these ideals, we get:
\bi
\item $rem_1 = f \xrightarrow[]{F_1\cup F_{0}}_+(\ga+1)a_2b_1b_2+(\ga^2+\ga)a_2b_2$
\item $rem_2 = f \xrightarrow[]{F_2\cup F_{0}}_+(\ga+1)a_2b_1b_2$
\item $rem_3 = f \xrightarrow[]{F_3\cup F_{0}}_+(\ga+1)a_2b_1b_2+a_2b_1 + (\ga^2+\ga)a_2b_2$
\item $rem_4 = f \xrightarrow[]{F_4\cup F_{0}}_+(\ga+1)a_2b_1b_2+a_2b_1$
\ei

When we compute $\prod_{l=1}^{2^m} 
rem_l\xrightarrow{J_{0}}_+$, 
 we obtain remainder 0, thus confirming
that the target set $W=(r_3,rr_3)$ indeed admits correction.
% Even though it is beyond the scope of this paper, $W=a_2b_1b_2+\be \cdot a_2b_2$ 
% is a polynomial which can be computed to rectify the circuit. 
% In fact, the rectification test also passes with nets $d_2$ and $d_5$;
% implying that we could have selected $w_0$ to be $d_2$ instead of $e_0$. 
% However, the rectification test fails when $w_0=d_0$ and
%  $w_1=d_5$. When the problem is formulated with these nets, $\prod_{l=1}^{4} 
%  rem_l\xrightarrow{J_{0}^{X_{PI}}}_+\al^9a_2b_1b_2+\al^{36}a_2b_2$  implying that
%  these nets do not admit correction.   
}
\end{Example}

% every $F_l\cup F_0$ forms a \Grobner basis in itself, and hence
% each corresponding $rem_l \in \Fkn[X_{PI}]$, and subsequently 
% $V(rem_l) \subseteq \F_{2}^{|X_{PI}|}$. 
% The authors in~\cite{Vkrao:ISQED21} limit their findings to 
% proving the {\it existence} of rectification functions by means of a check.
% However, on further investigation, one could
% use the information embedded within this check to 
% characterize the rectification functions.
The concepts presented in~\cite{Vkrao:ISQED21} are limited to
proving the {\it existence} of rectification functions.
However, our investigation further reveals that their result
can be extended to characterize the desired rectification functions.
Intuitively, the concept can be elaborated as follows.
The variety of $rem_l$ for any $l$ corresponds to the set of
all assignments to primary inputs $X_{PI}$ (minterms) where the
$\spec$ $f$ agrees with the $\impl$ $C$. Thus, the
condition of Thm.~\ref{Thm:rect} implies that the union of individual
varieties of $rem_l$'s comprises the set of all minterms where $f$ and $C$ evaluate the same. 
Thus, for every primary input assignment, {\it there exists} a cofactor tuple
assignment $W_c[l]$ to $W$ such that $f$ and $C$ match. Consequently, there
exists a set of functions $U = (u_1,\dots,u_m)$ that can be computed to 
rectify every error minterm. We exploit and explore this concept 
to compute rectification functions in the following section.


% To synthesize a patch, a desired $u_i$ corresponds
% to a polynomial function $u_i:\F_2^{|X_{PI}|} \mapsto \F_2$. 
% \begin{Proof}
% As the correction at target $W$ makes $C$ match $f$, $f$ should vanish on
% $V_{\Fkk}(J')$.
% Moreover, each $rem_l$ comprises only $X_{PI}$ variables. This is
% because WRTO $>_R$ ensures that each non primary input variable (each gate
% output and  word-level variable) appears as the leading term of some
% polynomial in $F'$. Thus each non primary input variable is canceled
% in the reduction $f\xrightarrow{F'_l, F'_{0}}_+ rem_l$. Furthermore,
% as $X_{PI}$ take values in $\F_2$, $x^2=x, \forall x \in
% X_{PI}$. Hence, 
% % even though $V_{\Fkk}(J')$ is evaluated in $\Fkk$,
% $V(rem_l) \subseteq \F_{2}^{|X_{PI}|}$. Thus, the rectification theorem
%  can be equivalently stated as: ``$f$ vanishes on
% \begin{small}
% $V_{\Fkk}(J') \iff \bigcup\limits_{l=1}^{2^m}V(rem_l) = \Ftwo^{|X_{PI}|}$''.
% \end{small} 

% (i) {\bf To prove ``$\Rightarrow$''}: Let $x_{PI} \in \Ftwo^{|X_{PI}|}$ be an
% assignment to the primary input variables of $C$. Every assignment
% $x_{PI}$ results in a corresponding assignment $x_{int}$ 
% to rest of the variables in $C$. For each such point $(x_{PI},x_{int})\in \Fkk$,
% the target $W$ evaluates to one of the values in the list $\delta$,
% i.e. $(0,1,\be,\dots,\beta^{2^m-2})$. When $W = 0$, $J'_1$ vanishes on
% the point $(x_{PI},x_{int})$. Likewise, $J'_2$ vanishes on
% $(x_{PI},x_{int})$ when $W = 1$, and so on. Since
% $f\xrightarrow{F'_l\cup F'_0}_+rem_l,1 \leq l \leq 2^m$, and $f$ vanishes
% on the point $(x_{PI},x_{int})$ to begin with, we obtain that for
% every  primary input assignment $x_{PI}$, one of the $rem_l$ vanishes. This
% implies that $ \bigcup\limits_{l=1}^{2^m}V(rem_l) = \Ftwo^{|X_{PI}|}$.

% (ii) {\bf To prove ``$\Leftarrow$''}: Say there exists an assignment to the
% primary inputs $x_{PI} \in \Ftwo^{|X_{PI}|}$ such that $rem_1$ vanishes on
% $x_{PI}$, i.e. $rem_1(x_{PI})=0$. For the given point $x_{PI}$, the rest of the variables 
% of $C$ get a corresponding assignment $x_{int}$. 
% As $f\xrightarrow{F'_1\cup F'_0}_+ rem_1$, we have that $f$ is a member of the
% ideal $J'_1 + J'_0 + \langle rem_1 \rangle$. Therefore, when
% $rem_1(x_{PI})=0$, the ideal $J'_1$ also vanishes on $(x_{PI},x_{int}) \in \Fkk$
% because the tuple $(x_{PI},x_{int})$ is a valid evaluation of the circuit.
% Further, $J'_0$ by definition vanishes everywhere in $R'$. This implies that
% $f(x_{PI},x_{int})=0$. The argument similarly holds for each
% $rem_{l}$ vanishing on some $x_{PI}$. This proves that for all primary
% inputs, if any $rem_l:1 \leq l \leq 2^m$ vanishes, then $f$ vanishes too; and 
% that completes the proof.
% \end{Proof}
 

%where the union of
%varieties corresponds to the product of ideals and is characterized by
%Strong Nullstellensatz over finite fields ({\red refer Strong
%  Nullstellensatz}). 
%% \begin{small}
%% \begin{align*}
%% &\bigcup\limits_{l=1}^{2^m}V(rem_l) =V_{\Ftwo}( \prod_{l=1}^{2^m}
%%   rem_l)= V_{\Ftwo}( \langle \prod_{l=1}^{2^m} rem_l \rangle +
%%   J_0^{X_{PI}} ) \\
%% & \quad\quad\quad\quad\quad\quad\quad\quad = V_{\Ftwo}(\langle \prod_{l=1}^{2^m} rem_l \rangle+ J_0^{PI})
%%   % V_{\Fqbar}(\langle r_1\cdot \dots r_{2^m} \rangle+ J_0^{PI}) \\
%% \end{align*}
%% \end{small}
%% Thus, to check for MFR at target $W$, we need
%% to check if $\prod_{l=1}^{2^m} rem_l\xrightarrow{J_{0}^{PI}}_+0$?  


% \begin{algorithm}\label{rect_flow_alg}
% \caption{Rectification of finite field arithmetic circuits}\label{pseudocode}
% \begin{algorithmic}[1]
% \Require $\spec:f, buggy~\impl: C$ modeled as a polynomial ideal $F=\{f_1\dots,f_s\}$ under $RTTO >$ 
% \Assume {$C$ doesn't admit single-fix rectification} //~\cite{Vkrao:FMCAD18}
% \Ensure {Rectification of $C$ to match $f$}
% \Procedure{$rectification$}{} 
% \State {remainder = $verify(f,F+F_0)$} // Sec.\ref{sec:verify}
% \State {$\Oa = analyze$(remainder)} //Sec. V.1~\cite{Vkrao:FMCAD18}
% \State {$\In = PotentialNets()$} //Sec.\ref{subsec:target_nets}
% \State {$m = 2$; rectified = $False; \mathcal{O}_A = \emptyset ; F_w = \emptyset$}
% \Do %1
% \State {$\Oa^i = uniquePartition(\Oa,m)$}\label{prtn}
% \If {$\Oa^i \notin \mathcal{O}_A$}
% \State {$\mathcal{O}_A=\mathcal{O}_A \cup \Oa^i$}
% \State {$\M^i = intersectionCover(\Oa^i)$}  
% \State {$f_w=pickTargets(\M^i,m)$}\label{ptrgt}
% \If {$f_w \notin F_w$}
% \State {$F_w=F_w \cup f_w$}
% \State {$MFRSetup()$} //Sec.\ref{subsec:comp_fwrk}
% \If {$MFRCheck(F')==0$} //Sec.\ref{sec:rcheck}
% \State {rectified = $True$}
% \State {patch = $rectFunction$()} //Sec.\ref{sec:rfunc}
% \Else
% \State {$goto~\ref{ptrgt}$}
% \EndIf
% \Else
% \State {$goto~\ref{prtn}$}
% \EndIf
% \Else
% \State $m++$
% \EndIf
% \doWhile {((!rectified) $\&\&$ $m \le |\Oa|$)} %1
% \State \Return patch
% \EndProcedure
% \end{algorithmic}
% \end{algorithm}

%  However, as discussed before, their variety 
% $V(rem_l)\subseteq \F_{2}^{|X_{PI}|}$. We can compute a polynomial $pr_l \in \F_{2}^{|X_{PI}|}$
%  such that $V(pr_l) = V(rem_l)$~\cite{Utkarsh:VLSI18}. 

% The authors discuss how rectifiability of $C$ against its 
% $\spec$ at a given set of $m$-distinct targets 
% $W=(w_1,\dots,w_m)\subset \{\{X\}-\{X_{PI}\}\}$ can be ascertained
% algebraically. 
%  