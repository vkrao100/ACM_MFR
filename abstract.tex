% Multi-Fix Rectification of buggy circuits requires deriving the necessary and 
% sufficient conditions for the existence of rectification functions at multiple nets, 
% and to subsequently compute the patch functions at these nets.  
% Our experiments show that the contemporary models which utilize SAT solvers, 
% fail to rectify finite field circuits beyond 16-bit operands.
% Utilizing a polynomial model and techniques from Symbolic Computer Algebra, 
% our approach rectifies buggy finite field arithmetic circuits with large operand widths. 

% Rectification of digital logic circuits with bugs often requires correction at 
% multiple targets. This process can be broadly divided into three main phases.
% First, identification of target signals for rectification. Second, deriving the necessary and 
% sufficient conditions for the existence of rectification functions at these targets. 
% Finally, computation of individual patch functions at these targets if the conditions are met. 
% In this paper, we assume that the targets are pre-specified and address the problem of 
% multi-target rectifiability check and subsequent computation of patch functions.
% Our experiments show that the contemporary models which utilize SAT solvers, fail to 
% rectify finite field circuits beyond 16-bit operands. Utilizing a polynomial model and 
% techniques from Symbolic Computer Algebra, we present a complete and scalable approach
% for multi-target rectification of buggy finite field arithmetic circuits and
% substantiate it with experimental results on large operand width benchmarks.
% Further, we also derive the observability don't care conditions for each of the 
% computed patches, thus providing a framework for logic synthesis and optimization.


\begin{abstract}
{\red This paper presents a symbolic computer algebra based approach for rectification 
of faulty finite field arithmetic circuits at multiple nets. Contemporary approaches
that utilize SAT solving and Craig interpolation are
infeasible in rectifying arithmetic circuits. 
Our approach employs a polynomial model and applies techniques from computer algebra 
to present a complete and scalable approach
for multi-target rectification of faulty finite field arithmetic circuits.
Given a set of $m$ nets as targets for rectification, first, we
utilize algebra-based techniques to ascertain existence of a rectification 
function at the targets, and subsequently compute a rectification function. 
Further, we show how the algebraic computing model allows to explore the
space of admissible rectification patches, collectively, for the $m$
targets. The model also enables exploration of don't care 
conditions across the $m$ targets for synthesis of rectification patches. 
In addition, we discuss the unate cube set interpretation of the polynomial 
model and how the unate cube set algebra prowess of ZDDs can be
harnessed to efficiently represent and manipulate the model.
We substantiate the approach with experimental results 
demonstrated over large operand width benchmarks used in cryptography applications.}
\end{abstract}