% Multi-Fix Rectification of buggy circuits requires deriving the necessary and 
% sufficient conditions for the existence of rectification functions at multiple nets, 
% and to subsequently compute the patch functions at these nets.  
% Our experiments show that the contemporary models which utilize SAT solvers, 
% fail to rectify finite field circuits beyond 16-bit operands.
% Utilizing a polynomial model and techniques from Symbolic Computer Algebra, 
% our approach rectifies buggy finite field arithmetic circuits with large operand widths. 

% Rectification of digital logic circuits with bugs often requires correction at 
% multiple targets. This process can be broadly divided into three main phases.
% First, identification of target signals for rectification. Second, deriving the necessary and 
% sufficient conditions for the existence of rectification functions at these targets. 
% Finally, computation of individual patch functions at these targets if the conditions are met. 
% In this paper, we assume that the targets are pre-specified and address the problem of 
% multi-target rectifiability check and subsequent computation of patch functions.
% Our experiments show that the contemporary models which utilize SAT solvers, fail to 
% rectify finite field circuits beyond 16-bit operands. Utilizing a polynomial model and 
% techniques from Symbolic Computer Algebra, we present a complete and scalable approach
% for multi-target rectification of buggy finite field arithmetic circuits and
% substantiate it with experimental results on large operand width benchmarks.
% Further, we also derive the observability don't care conditions for each of the 
% computed patches, thus providing a framework for logic synthesis and optimization.


\begin{abstract}

This article presents a complete and scalable symbolic computer algebra 
approach for rectification of faulty finite field arithmetic circuits 
at multiple nets. Contemporary approaches
that utilize SAT solving, and Craig interpolation is
infeasible in rectifying arithmetic circuits. 
Our approach represents the circuit as a system of polynomials and rectifies it against a polynomial specification by applying \Grobner basis (GB) based algorithms. 
Given a set of $m$ candidate nets as rectification targets, first, we
utilize algebra-based techniques to derive the necessary and sufficient conditions
for the existence of a rectification function at the targets. Then, upon confirmation, 
we compute the patch functions collectively for the targets with variable support 
in primary inputs.  
For patch function computation, we present two approaches: a greedy approach which
resolves the rectification functions for the targets, and an approach 
which explores a subset of don't care conditions for the targets.
In this regard, we show how the algebraic computing model allows to explore the
space of admissible rectification patches, collectively, for the $m$
targets. Our approach is implemented as a custom software and utilizes 
existing open-source symbolic algebra libraries for computations. 
The core GB-based reduction computation on circuits is performed using the Boolean 
data-structure of Zero-suppressed Binary Decision Diagrams (ZDDs).   
We substantiate the approach with experimental results 
demonstrated over large operand width benchmarks up to 571 bits, 
including those that conform to the NIST-standard for ellyptic curve cryptography.


% We show how the unate cube set algebra prowess of ZDDs can be
% harnessed to represent and manipulate the model efficiently for rectification.
% Further, we show how the algebraic computing model allows to explore the
% space of admissible rectification patches, collectively, for the $m$
% targets. The model also enables exploration of don't care 
% conditions across the $m$ targets for synthesis of rectification patches. 

\end{abstract}