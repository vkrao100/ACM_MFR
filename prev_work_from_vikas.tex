
% Introducing SCA for verification
Symbolic Computer Algebra (SCA) techniques are found to be more
suitable for formal analysis and verification of arithmetic circuits.
However, utilization of the various facets and capabilities of the SCA
techniques for post-verification debugging and rectification has only
recently begun to be addressed~\cite{farimah:2017:1,MF_Rolf:ISVLSI18,Utkarsh:VLSI18,
Vkrao:FMCAD18,Utkarsh:ETS19,Vkrao:ISQED21}. }

% Latest prior work related to rectification 
Contemporary approaches formulate rectification
using QBF solving~\cite{scholl:2}, using CI or iterative SAT
solving~\cite{MF_Roland:ICCAD10,MF_Huang:DATE12}.  
The rectification techniques in~\cite{MF_Huang:DATE12,SS_Roland:DAC18} iteratively 
and incrementally compute multiple single-fix functions that partially patch the 
circuit in each iteration. The more recent techniques further include more resource 
awareness in patch generation by reusing existing logic~\cite{SS_Alan:DAC18}, 
employing improved heuristics for target selection~\cite{SS_Fujita:ISCAS19}, 
adding reconfigurable components to scale the problem at a higher abstraction level
~\cite{SS_Fujita:ISCAS19_2}, or enumerate rectification points and their drivers 
by a combination of simulation and rewiring strategy~\cite{SS_Roland:DAC19}. 
Interested readers can refer~\cite{SS_Roland:DATE20} for a comprehensive analysis
of the contemporary rectification techniques, their limitations, and future direction. 

%%%%%     Include the below paragraph if required - or can cut down if short on space

% In~\cite{SS_Roland:DAC19} the authors propose a robust ECO approach to derive 
% patches with minimal impact on the heavily optimized existing implementation 
% against a structurally dissimilar ECO-evolved specification. 
% They enumerate rectification points 
% functionally by simulation and match the circuitry of patches implicitly 
% to maximize reuse of existing logic in the implementation. To achieve 
% scalability, the method proposes modeling and analyzing its computations 
% in symbolic sampling domain. 


However, circuits that implement polynomial 
computations over large bit-vector operands are hard to rectify using 
models based on Boolean function, SAT/SMT-solvers, etc.

